\subsection{Lexicogrammatical choice}
\label{sec:LexicogrammaticalChoice}

Other pauses and online revisions appear to be motivated by lexicogrammatical choice. However, some of them do not occur in grammatical boundaries, not even when taking bound morphemes into account. The motivation for the pauses, as we shall see next, seems to be at the graphological stratum, namely at the comparison between character sequences. If you know German, take your time to read Examples \ref{ex:6}-\ref{ex:10} before going on with the reading. Make your own conjectures and contrast them with mine.

\begin{exe} % eine ganz andere ?
  \ex\label{ex:6}$\cdot$\={ }g\={ }ä\u{ }n\u{ }z\={ }\uettl\u{ }\uettl\u{ }\uettl\={ }ä\u{ }n\u{ }z\u{ }l\u{ }i\u{ }c\u{ }h\u{ }$\cdot$
\end{exe}

\begin{exe} % eine ganz andere ?
  \ex\label{ex:7}$\cdot$\={ }e\u{ }i\u{ }n\u{ }e\u{ }s\u{ } \={ }s\u{ }o\={ }l\u{ }c\u{ }h\u{ }e\u{ }n\u{ }$\cdot$
\end{exe}

\begin{exe} % Verhandlungsweise, Verhandensweise, Verfahrensweise ?
  \ex\label{ex:8}$\cdot$\u{ }D\u{ }i\u{ }e\u{ }$\cdot$\={ }\uettl\u{ }\uettl\u{ }\uettl\u{ }a\u{ }s\u{ } \u{ }V\u{ }e\u{ }r\u{ }h\u{ }a\u{ }l\u{ }t\u{ }e\u{ }n\u{ }$\cdot$
\end{exe}

\begin{exe} % Verhandlungsweise, Verhandensweise, Verfahrensweise ?
  \ex\label{ex:9}$\cdot$\={ }d\={ }e\u{ }r\={ }\uettl\u{ }s\u{ }$\cdot$\u{ }V\u{ }e\u{ }r\={ }h\u{ }a\u{ }l\u{ }t\u{ }e\u{ }n\u{ }s\={ }$\cdot$
\end{exe}

\begin{exe}
  \ex\label{ex:10}$\cdot$\={ }w\={ }e\u{ }i\u{ }t\={ }e\u{ }r\u{ }e\={ }$\cdot$\={ }\uettl\={ }r\u{ }$\cdot$\={ }K\u{ }o\u{ }m\u{ }p\u{ }r\u{ }e\u{ }s\u{ }s\u{ }i\u{ }o\u{ }n\u{ }$\cdot$\={ }z\u{ }u\u{ }$\cdot$\={ }w\u{ }i\u{ }e\u{ }d\u{ }e\u{ }r\u{ }s\u{ }t\u{ }e\u{ }h\u{ }e\u{ }n\={ }$\cdot$
\end{exe}

In Examples \ref{ex:6} and \ref{ex:7}, we see a very interesting pause and erase pattern. I suspect the translator was unsure whether to write the most frequent \emph{eine \underline{ganz} andere Geschichte} (\emph{a `whole' different story}) and \emph{\underline{so} eines Balls} (`this kind of' ball) or the less frequent and register-specific variants \emph{eine \underline{gänzlich} andere Geschichte} (\emph{a `completely' different story}) and  \emph{eines \underline{solchen} Balls} (\emph{`such a' ball}). The overlap between the underlined strings is \emph{\underline{g}ä\underline{nz}lich} and \emph{\underline{so}lchen}. Coincidentally or not, the translator paused once at the end of each overlap and, in Example \ref{ex:6}, he or she erased the left characters up to the end of the first overlap, where he or she could finish the word either as \emph{g\underline{anz}} or as \emph{g\underline{änzlich}} and ended up choosing \emph{gänzlich} and writing it without pauses until the end. Are these pauses motivated by lexical choice? I would say so.\footnote{6 other translators chose \emph{eine ganz andere Sache}, 1 \emph{eine ganz andere Frage}, 1 \emph{eine ganz andere Angelegenheit}, 4 \emph{eine komplett andere Sache}, 1 \emph{doch eine andere Angelegenheit}. This indicates that the deictic modifier \emph{ganz} is the least marked one for this co-text, followed by \emph{komplett}, followed by the once produced modifiers \emph{gänzlich} and \emph{doch}.}.
But are their locations motivated by word boundary? I would say no. Comparison of graphological classes of words (comparison of string) seems to be the motivation.

Moreover, in logs of writing actions we find not only direct replacements of lexical words, but also indirect clues that such a replacement took place in an underlying cognitive process. It seems to be the case that the translator first produces a segment of the target text cognitively and then writes this cognitive segment down. While writing the segment down, it seems to be the case that the translator continues the production of the target text and, depending on what comes, he or she needs to change parts this text that were already written down.

Examples \ref{ex:8} and \ref{ex:9} are evidences that such a process might go on. In both cases a `feminine' Deictic word\footnote{Following the tradition of Systemic Functional Linguistics, the contextual function of words is capitalised. For instance, the word \emph{selbe} in \emph{dieselbe rote Regenjacke} `the same red rain jacket' is as much an adjective as the word \emph{rote} because both of them are inflected in the same way. However, \emph{selbe} works as a Deictic\textsubscript{2} because it relates the mentioned jacket with previously mentioned or previously observed jackets whereas \emph{rote} works as a Classifier because it adds a color restriction for discriminating the jacket. Meanwhile, the word \emph{rain} also works as a Classifier because it adds a functional restriction for discriminating the jacket. Despite that fact, it is not an adjective itself. In that sense, Deictic, Classifier and Thing are functions of words and not inflectional classes of words such as determiner, adjective and noun.}, namely \emph{Die} and \emph{der}, was replaced by a `neutral' Deictic word, namely \emph{Das} and \emph{des}. In German, Deictic words typically agree with the Thing word in grammatical gender: masculine \emph{der}/\emph{den}/\emph{dem}/\emph{des}, feminine \emph{die}/\emph{die}/\emph{der}/\emph{der}, and neutral \emph{das}/\emph{das}/\emph{dem}/\emph{des}. I looked up in a synonyms dictionary what could be alternative `feminine' Thing words for \emph{Das Verhalten} and \emph{des Verhaltens}. There were many. However, since the translator made a pause after \emph{Ver} in \emph{Verhalten}, I assumed the lexical item he or she was considering might start with \emph{Ver} and continue with a different letter than \emph{h}. Then I listed all `feminine' alternatives starting with \emph{Ver} and reached the following list: \emph{Verhaltungsweise}, \emph{Verhaltensweise}, and \emph{Verfahrensweise}. Only \emph{Verfahrensweise} has a letter different from `h' following \emph{Ver}. So, if the assumptions that the translator revisited his/her lexical choice and that he/she stopped at that point because of the string overlap are right, the other lexical item considered for that position might be \emph{Verfahrensweise} as in \emph{Die Verfahrensweise} and \emph{der Verfahrensweise}.
% TODO No need to put it this pessimisticly: It is a valid hypothesis which needs to be confirmed by means of on-line or retrospective verbalisation. Full stop.
Such a claim has no scientific validity at the current stage, but being able to make such hypotheses might be helpful. Researchers can ask the translator right after the translation whether this was indeed a lexical choice they considered. It might be the case translators are able to recall what they considered at that point in time.

When looking at these examples, one might assume that only adjacent words or Deictic words within a nominal or verbal group such as \emph{\underline{das} Verhalten} (the behaviour) and \emph{\underline{die} Verfahrensweise} (the behaviour) or such as \emph{\underline{hat} sich so verhalten} (behaved in this way) and \emph{\underline{ist} so verfahren} (behaved in this way) would be subject to such changes. Example \ref{ex:10} shows that neither the notion of adjacency nor the notion of co-constituents is sufficient for explaining such phenomena. In this case, the translator had three gender options and four case options for the nominal group. Given the replacement of \emph{weitere} (further) by \emph{weiterer} (further), I suspect that this was a choice between accusative and dative cases for the feminine gender. For this hypothesis, the translator considered the options of feminine accusative \emph{weitere Kompression} (\emph{further compression}) and of feminine dative \emph{weitere\underline{r} Kompression} (\emph{further compression}). The combination of a fixed gender with a variable case would imply that the lexical item \emph{Kompression} (\emph{compression}) was already selected for the nominal group and that the lexical item for the verbal group \emph{etwas widerstehen} (\emph{to resist to something}) was not.

This hypothesis is very interesting from a linguist's perspective. A `compression' is not a physical thing. It is rather something that we would rather call an ongoing process. Such `processual things' often have the role of Scope in a material figure \citep[p. 192]{Halliday:2004fe} and are typically represented in clauses with the following participant role sequence: Agent + Process + Scope as in \emph{{[}das{]} {[}tut{]} {[}weitere Kompression{]}} (\emph{this does further Kompression}), \emph{{[}das{]} {[}macht{]} {[}weitere Kompression{]}} (\emph{this makes further Kompression}), \emph{{[}das{]} {[}verursacht{]} {[}weitere Kompression{]}} (\emph{this causes further Kompression}). The nominal group representing the processual thing and functioning as Scope is typically accusative, what justifies the default choice for accusative by the translator. However, the lexical item for the verbal group did not represent a process of doing, making or causing something, i.e. making something happen. It represented a process of resisting to something, acting against some external force so that nothing happens. In German, the lexical item \emph{etwas widerstehen} (\emph{to resist to something}) happens in clauses such as \emph{{[}so ein Ball{]} {[}widersteht{]} {[}weitere\underline{r} Kompression{]}} (\emph{such a ball resists to further Kompression}), which have a Scope Complement constituent that is a dative nominal group. Based on that, when the translator reached the `critical' point for case selection, having decided that the next lexical item is \emph{Kompression} (\emph{compression}) was not sufficient. He or she are likely to have selected the lexical item \emph{etwas widerstehen} (\emph{to resist to something}) at this point for avoiding a time-consuming online revision if this decision were to be postponed.

Moving on, it must have become clearer by this point that some typing pauses seem to be motivated by lexical item choice, but that simultaneously these pauses are not necessarily placed at the boundaries of grammatical constituents such as morphemes, words, groups, phrases, and clauses. They are oftentimes placed at the borders of overlapping character sequences among two or more considered graphological classes of words.

Whether those lexicogrammatical feature selection is a bilingual intellectual process is still open to debate. In my opinion, none of these revisions are necessarily supported by bilingual processes and I can imagine they might also happen when writing an original text from scratch. Maybe these revisions might be more frequent in one activity than in the other. I do not have evidence for sustaining any claim in this or that direction.