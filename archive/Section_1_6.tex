\subsection{Micro/macro units of translation}
\label{sec:MicroMacro}

Online revisions are just one kind of revision.  Since text-replacing writing action are a process of replacing a text by another, I shall consider any sequence of text-replacing writing actions a \textbf{revision} at the graphic and graphological strata. Also assuming that a grammatical structure -- namely, a non-random source text segment -- is put under translation at a time, \citet{Alves:2009js,Alves:2011vj} defined the notion of micro-unit of translation: the span of writing activity that produces a target text segment that is equivalent to a source text segment under translation.
% TODO this definition should occur earlier
The first span of writing actions for a given source text segment equivalent was understood as a segment insertion (P0) and the replacements of that segment by other source text segment equivalents was understood as a segment replacement. The first replacement was classified as P1 if it happened in the drafting phase and it was classified as P2 if it happened in the revision phase. The second, third and following replacements of equivalents of the same source text segment were classified as P3. Finally, a sequence of text revisions that affects equivalents of the same segment of the source text was conceived of as a macro-unit. A macro-unit is composed of one or more revisions: it contains necessarily a P0 revision, which is either final or followed by a P1 or P2 revision, which is either final or followed by one or more P3 revisions. `P' here stands for `process unit'.

Whereas online replacements (P1 or P3) can be easily understood just by looking at a sequence of writing actions, the understanding of revision phase replacements (P2 or P3) depends strongly on the reconstructed text and on the position of the cursor during erase actions and char insert actions. In the log of writing actions, they look like this: \={ }\uettl\={ }\uettl\u{ }D\={ }.

However, there is nothing special about those events in the nature of replacements as far as what they actually do to the target text, except that, as \citet{Alves:2011vj} point out, translators seem not to go back to the source text so often during the revision phase. Therefore, the reasons for replacements during that phase are even less directly motivated by the source text and possibly not supported by any bilingual intellectual process. This means that, as the translation moves from P0 to P3, chances are that the translator thinks progressively less bilingually and progressively more monolingually.

Taking this into account, it is important to notice that the very notion that supports such a micro-unit and a macro-unit rationale is a correspondence or an alignment between source and target lexicogrammatical structures. This is the very assumption that makes us researchers in translation studies want to take process units as evidence for anything in the process of translation. But are we misguided in taking the micro-units, as \citet{Alves:2009js,Alves:2011vj} call them, to be \textbf{\textit{any}} span of writing actions between pauses in typing of a given length? I am afraid we are. If the amount of grammatically unrelated phenomena that motivates pauses were not enough evidence, let us consider the following example:

\begin{exe}
  \ex\label{ex:11}\={ }d\u{ }i\u{ }e\u{ }$\cdot$\={ }K\={ }r\u{ }a\u{ }f\u{ }t\={ }\uettl\u{ }\uettl\u{ }\uettl\u{ }\uettl\u{ }\uettl\={ }a\u{ }u\u{ }f\u{ }g\u{ }e\u{ }w\u{ }e\u{ }n\u{ }d\u{ }e\={ }t\u{ }e\u{ } \u{ }K\={ }f\={ }\uettl\u{ }r\u{ }a\u{ }f\u{ }t\u{ }$\cdot$\={ }z\={ }u\={ }\\ 
\uettl\={ }\uettl\={ }\uettl\u{ }\uettl\u{ }\uettl\u{ }\uettl\u{ }\uettl\u{ }\uettl\u{ }\uettl\={ }$\cdot$\u{ }K\u{ }o\u{ }m\u{ }p\u{ }r\u{ }e\u{ }s\u{ }s\u{ }i\u{ }o\u{ }n\u{ }s\u{ }k\u{ }r\u{ }a\u{ }f\u{ }t\u{ }$\cdot$
\end{exe}

\begin{table}[h!]
\centering
\begin{tabular}{ | p{0.22\linewidth} | p{0.44\linewidth} | }
\hline
text version & character sequence segment \\
\hline
0 & \\
1 & die$\cdot$ \\
2 & die$\cdot$K \\
3 & die$\cdot$Kraft \\
4 & die$\cdot$ \\
5 & die$\cdot$aufgewende \\
6 & die$\cdot$aufgewendete$\cdot$K \\
7 & die$\cdot$aufgewendete$\cdot$Kf \\
8 & die$\cdot$aufgewendete$\cdot$K \\
9 & die$\cdot$aufgewendete$\cdot$Kraft$\cdot$ \\
10 & die$\cdot$aufgewendete$\cdot$Kraft$\cdot$z \\
11 & die$\cdot$aufgewendete$\cdot$Kraft$\cdot$zu \\
12 & die$\cdot$aufgewendete$\cdot$Kraft$\cdot$z \\
13 & die$\cdot$aufgewendete$\cdot$Kraft$\cdot$ \\
14 & die$\cdot$aufgewendete \\
15 & die$\cdot$aufgewendete$\cdot$Kompressionskraft$\cdot$ \\
\hline
\end{tabular}
\caption{ Target text segments during each pause of Example \ref{ex:11} }
\label{tab:ActionsEx11}
\end{table}

In Table \ref{tab:ActionsEx11}, when considering replacements at the grammatical stratum, it seems reasonable to imagine that the nominal group had three versions. The first version is completed at text version 3, thus resulting the P0 minor-unit of translation 0-3, the second version of the nominal group would be complete at text version 9, thus resulting the P1 minor-unit of translation 3-9, an incomplete attempt would end at version 11, resulting the P3 minor-unit of translation 9-11, and, finally, the span from 11-15 would be a fourth minor-unit of translation at the grammatical level. We were able to reach this chunking of the process not based on pauses, but based on the grammaticality of character sequences as products of writing.

At the same time, if we take another criterion such as key-moving actions, we find other `minor units' (this time not of translation). This time, we can explain the pause before the \textbf{`r' key down} action at versions 2 and 6 as being possibly due to the translators bad left hand position, and the replacement of the character `f' by the character `r' -- spanning from version 7-9 -- as being motivated by an erroneous \textbf{key down} action, due to a possibly bad left hand position. In the typing process, the span from 6 to 7 would be a P0 revision and the span from 7 to 9 a P1 revision: the first span inserts the character `f' and the second span replaces the character `f' by the character `r'. In parallel to this, there is probably another process going on. The lexical item \emph{aufgewendete} is an alternative to the lexical item \emph{aufgewandte}, the overlap of the written word with the alternative is \emph{\underline{aufgew}e\underline{nde}te}. At the end of the overlap, there is a pause. Did the translator reconsider which lexical item to choose at this point? This might well be the case. Finally, the partially written character sequence \emph{zu...} might have been completed as \emph{zur Kompression} (\emph{in the compression}) as in \emph{die aufgewendete Kraft zur Kompression} (\emph{the force spent in the compression}) or as \emph{zu komprimieren} (\emph{in compressing}) as in \emph{die aufgewendete Kraft zu komprimieren} (\emph{the force spent in compressing}). Such target text segments were likely discarded and a new one was typed until the end \emph{die aufgewendete Kompressionskraft} (\emph{the spent compression force}). This kind of replacement is different from the one of replacing \emph{die Kraft} by \emph{die aufgewendete Kraft}. The earlier revision is an insertion of a word before another word that was already written. The later revision is indeed a revision of a way of formulating to another. In that case, if we go down from the nominal group to the constituents of the nominal group, what looked like four minor-units of translation becomes one P0 minor unit of translation for \emph{aufgewendete$\cdot${[}Kraft$\cdot${]}} (\emph{spent$\cdot${[}force$\cdot${]}}) and two minor units of translation, \emph{{[}Kraft$\cdot${]}zu{[}$r\cdot$Kompression$\cdot${]}} | \emph{Kompressions{[}kraft$\cdot${]}} (\emph{{[}force$\cdot${]}in$\cdot$the$\cdot$compression} | \emph{compression$\cdot${[}force$\cdot${]}}.

In other words, it is a selection of the process (motion, typing, writing, saying), the stratum (graphics, graphology, lexicogrammar, and rhetoricosemantics), and the rank (morpheme, word, group, phrase, clause) that makes the detection of micro-units and macro-units possible. Pauses do indeed help us in finding out whether there is more or less effort at a particular point in writing. But the reasons why a pause is there is multivariate. Next, I shall review how pauses in typing have been calculated so far and suggest a heuristics to estimate a pause in typing. With such a heuristics we shall be able to avoid relying on pauses in writing as has been the praxis so far.