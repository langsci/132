\subsection{Erase actions}
\label{sec:EraseActions}

In addition to the assumption of translation boundaries at pauses, another problematic assumption is that online revisions, i.e. revisions during the drafting phase, are related to change in the choice of semantic features. Some of these pauses indeed seem to be semantically motivated whereas for many others such an interpretation seems questionable. In the following examples, the symbol \uettl{ } indicates a \textbf{left char erase} action, where \textbf{left char} is the character left of the cursor.

% Key hitting issues | Motoric issues

% Typing the adjacent character
\begin{exe}
  \ex\label{ex:4a}$\cdot$\u{ }K\={ }f\=\uettl\u{ }r\u{ }a\u{ }f\u{ }t\u{ }$\cdot$ (Translator 2)
\end{exe}

% Wrong order
\begin{exe} 
  \ex\label{ex:4b}$\cdot$\={ }k\u{ }e\u{ }i\u{ }e\u{ }n\={ }\uettl\u{ }s\u{ } \={ }\uettl\u{ }\uettl\u{ }\uettl\={ }n\u{ }e\u{ }s\u{ }$\cdot$ (Translator 2)
\end{exe}

% Typing issues | Layout issues

% Typing two characters
\begin{exe} 
  \ex\label{ex:4c}$\cdot$\={ }b\u{ }v\={ }\uettl\={ }e\u{ }s\u{ }t\u{ }e\u{ }h\u{ }t\u{ }$\cdot$ (Translator 2)
\end{exe}

In Example \ref{ex:4a}, what might have happened is something in the following lines. The translator had the right hand well positioned on the keyboard and the left hand somewhat misplaced and was aware of it. When typing, the left shift key was still under the left hand and was quickly moved down, what was followed by a \textbf{`k' key down} action with the right hand in regular speed. However, after inserting the `k' character, the time taken to insert the next character, namely `f', is longer than usual. Here, the translator might have had the need to reposition his/her left hand and still accidentally moved down the `f' key instead of the `r' key. In the German keyboard used in the experiment, the `f' key is directly below the `r' key and a badly positioned left hand is a sufficient reason for a `typo'.  After inserting `f' instead of `r', there is another long pause\footnote{It is definitely the case that some online revision are initiated by a production failure being detected in quality monitoring processes. Missed keys are an example of such cases since it is by comparing the intended text and the produced one that such mistakes (typos) are likely to be detected and fixed.}, which is followed by a \textbf{left char erase} action. Such pauses and such an erase action do not seem to have anything to do with any bilingual process.

In Example \ref{ex:4b}, something else happens. The translator ends up writing the word \emph{keines}, but in the first run of char insert actions, he or she writes \emph{keien} instead of \emph{keine}. The issue here was not one of moving the wrong keys, but one of moving the right keys in the wrong order. A good point to notice here is that the \textbf{`e' key down} action is usually performed with the left hand with a German keyboard whereas the \textbf{`n' key down} action is usually performed with the right hand. This means that this mistaken order happened when coordinating the motion of both hands. The correction procedure is quite interesting too. Only part of the problem is solved by the first attempt, resulting \emph{keies} instead of \emph{keines}, this is again noticed and the second revision procedure leads to the version \emph{keines}.

Example \ref{ex:4c} is not so simple. Here the translator ends up writing \emph{besteht}, but types very quickly and with the same hand two letters. In the used German layout, the \textbf{`v' key} and the \textbf{`b' key} are adjacent and are both right of the index finger (the finger a translator would move these keys down with). Was it that the translator was unsure which key was the `b' key and moved both down in a sequence to decide which character to keep on screen or was it that the translator just moved both keys down accidentally? I do not have an answer for such a question. But one thing is for sure, this was not a bilingual process, not even a monolingual process in the sense of choosing what to say.

Furthermore, there is another type of revision that seems to be related to the translation process in a rather `non-linguistic' way. Such revisions are also not graphological, not lexical, not grammatical, and not semantic. Examples \ref{ex:4d} and \ref{ex:4e} illustrate them. In such cases, the translator seems to copy the source text instead of writing a target text. In Example \ref{ex:4d}, the character sequence under translation is \emph{demands}, the target character sequence is \emph{bedarf}. Grammatically valid alternatives in German could have been \emph{bedürfte}, \emph{benötigt}, \emph{braucht}, and \emph{bräuchte} but no word starting with \emph{d}. Example \ref{ex:4e} shows a similar phenomenon: the source character sequence is \emph{paper} and the target character sequence is \emph{Papier}. However, the translator writes \emph{Pape}. Is it the case that he or she just missed typing the \textbf{`i' key} or is it the case that he or she copied the source character sequence instead of writing the target one? Again, I do not have an answer for this, but this does not seem to be a bilingual process in the typical sense of what we understand by translation\footnote{This source text copying does not seem to be an effect of `priming' given that the translator is unlikely to accept the source language character sequence as a valid word in the target language. However, the notion of priming could be applied to other examples if translators choose a marked lexical item instead of a less marked one because of string similarity or shared etymological origin of the lexical items in both languages.}\footnote{As for any other online revision preceded by a breve typing pause, a quality-monitoring process can be inferred for such cases.}.

% Task issues | Copying ?
\begin{exe} 
  \ex\label{ex:4d}$\cdot$\u{ }d\u{ }\uettl\u{ }b\u{ }e\u{ }d\u{ }a\u{ }r\u{ }f\={ }$\cdot$  (Translator 2)
\end{exe}

\begin{exe} 
  \ex\label{ex:4e}P\u{ }a\u{ }p\u{ }e\u{ }\uettl\={ }i\u{ }e\u{ }r\u{ }$\cdot$ (Translator 2 : 1st `Papier')
\end{exe}

% TODO Is this a different "Papier" or a different translator? Could you please report on the data in greater detail?

Other revisions such as Examples \ref{ex:4} and \ref{ex:5} look more linguistic. However, they are also not grammatical: the first is a replacement of a \emph{latin small letter P} by a \emph{latin capital letter P} and the second is a replacement of \emph{latin capital letter G} by a \emph{latin small letter G}. As seen before, replacements of characters are not performed necessarily with one \textbf{left char erase} action followed by one \textbf{char insert}. It may take more than two writing actions for realising the replacement of one character: in Example \ref{ex:5}, a total of eight writing actions were performed for changing one letter from capital to small. Tables \ref{tab:ActionsEx4} and \ref{tab:ActionsEx5} show the series of resulting texts for both examples.

\begin{exe}
  \ex\label{ex:4}$\cdot$\={ }p\={ }\uettl\u{ }P\u{ }a\u{ }p\u{ }i\u{ }e\u{ }r\u{ }$\cdot$ (Translator 11 : 6th `Papier')
\end{exe}

\begin{table}[h!]
\centering
\begin{tabular}{ | p{0.22\linewidth} | p{0.22\linewidth} | }
\hline
writing & resulting text \\
\hline
\hline
U+0070 char insert & p| \\
\hline
\hline
left char erase & | \\
\hline
U+0050 char insert & P| \\
\hline
U+0061 char insert & Pa| \\
\hline
U+0070 char insert & Pap| \\
\hline
U+0069 char insert & Papi| \\
\hline
U+0065 char insert & Papie| \\
\hline
U+0072 char insert & Papier| \\
\hline
\end{tabular}
\caption{Writing actions of Example \ref{ex:4}}
\label{tab:ActionsEx4}
\end{table}

\begin{exe}
  \ex\label{ex:5}$\cdot$\u{ }G\u{ }r\u{ }o\={ }\uettl\u{ }\uettl\u{ }g\u{ }\uettl\u{ }\uettl\u{ }g\u{ }r\={ }o\u{ }ß\u{ }e\u{ }n\u{ }$\cdot$
\end{exe}

\begin{table}[h!]
\centering
\begin{tabular}{ | p{0.22\linewidth} | p{0.22\linewidth} | }
\hline
writing & resulting text \\
\hline
\hline
U+0047 char insert & G| \\
\hline
U+0072 char insert & Gr| \\
\hline
U+006F char insert & Gro| \\
\hline
\hline
left char erase & Gr| \\
\hline
left char erase & G| \\
\hline
U+0067 char insert & Gg| \\
\hline
left char erase & G| \\
\hline
left char erase & | \\
\hline
U+0067 char insert & g| \\
\hline
U+0072 char insert & gr| \\
\hline
\hline
U+006F char insert & gro| \\
\hline
U+00DF char insert & groß| \\
\hline
U+0065 char insert & große| \\
\hline
U+006E char insert & großen| \\
\hline
\end{tabular}
\caption{Writing actions of Example \ref{ex:5}}
\label{tab:ActionsEx5}
\end{table}

As far as translation studies are concerned, such online character replacements are not very interesting and pauses related to them, independent of how long they are, should not be assumed to be motivated by the boundaries of grammatical structures.