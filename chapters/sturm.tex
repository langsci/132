\documentclass[output=paper]{LSP/langsci} 
\ChapterDOI{10.5281/zenodo.1090990}
\author{Annegret Sturm
\affiliation{University of Geneva}
}
\title{Metaminds: Using metarepresentation to model minds in translation}

\abstract{Addressing the other is fundamental to translation studies. Language is the unique human capacity for interaction by transferring meaning, emotions and attitudes to another mind. The translator has to understand the auther's intentions behind the communication in order to correctly interpret and adapt her message for the target audience. One of the most interesting features of translation is this double metarepresentation of author and audience. 

The aim of this paper is (1) to conceptualise translation as higher-order metarepresentation and (2) to show empirically that the permanent taking and giving of other's perspectives shapes the translator's mind. 

I shall begin with outlining why translation is an intensive mental interaction, and how previous literature has dealt with the translator's mental interaction with the two others. After introducing the concept of attributive metacognition, or Theory of Mind (ToM), I shall review the literature on how translation trains attributive metacognition. If translation really is such a highly demanding task in terms of attributive metacognition, translators should have a better ToM than non-translators. I set up an fMRI experiment to study this question.

The results show an important activation in the precuneus for both groups. Labelled as ``the mind's eye'' \citep{Fletcher1995}, the precuneus is the region that subserves the representation of the self in relationship with the outside world \citep{Cavanna2006} as well as perspectives contrary to our own \citep{Bruneau2010}.}

\maketitle

\begin{document}

\section{Introduction}

From the outside, translation seems to be a rather lonely activity: The translator interacts with text and hardly ever with other individuals. Bizarrely, however, translators experience their work as highly interactional. They call their activity ``an act of supreme empathy'' (Simic, in \citet[107]{Kelly2012}, ``an act of love'' (Steiner, in \citealt[213]{Kelly2012}) and ``a valuable way of coming closer'' \citep[119]{Bassnett2002}. These individual experiences hint at the hidden character of translation as social activity on the mental meta-level.  


The following sections present different models of translation to show that translation has always been considered a phenomenon on the meta-level, be it textual or communicative. The main point of this paper is to extend this view to include metacognition and in particular, to consider competences in \isi{attributive metacognition} as one of the core components of \isi{translation competence}.


Translation involves many competences at the meta-level \citep{Plassard2007}. In the late 1970ies, translation scholars started to think of translation as a metatext \citep{Popovic1976}. As a textual reaction to prior text, translation was similar to reader's letters to the editor \citep[232]{Popovic1976}. The translator is thus a reader who reacts to prototext. But in contrast to other readers, the translator's reaction is a reproduction of the original. The communicative impact of the newly created metatext , however, depends entirely on the reader's \isi{frame} of reference \citep[230]{Popovic1976}. As a reader who recreates the text s/he has just perceived for other readers, the translator's capacity to anticipate their frames of reference is crucial for the communication. 


Indeed, translation was soon to be considered as an act of communication instead of a textual genre.  In \citeyear{Bigelow1978}, John Bigelow conceptualised translation as a form of indirect speech. Ever since, this is one of the most frequently used approaches to model the \isi{translation process}. The language switch becomes a secondary and not necessarily defining feature of translation. For the philosopher Donald Davidson, sameness of meaning exists independent of language on the level of the language user \citep[125]{Davidson2010}. Since every language provides means for indirect speech, every language user must have the cognitive possibilities for \isi{interpreting} indirect communication. Communication across languages is hence possible because everybody who can understand and produce monolingual indirect communication has the necessary cognitive means for understanding and producing multilingual indirect communication.


Translation is possible because it does not require any special mental equipment that would not be used in inferential communication in general \citep[200]{Gutt2000}. But then what makes translation different from monolingual standard communication, if not the language switch? According to Relevance Theory, standard inferential communication is characterised by the so-called mutual cognitive environment, i.e. shared information between the interlocutors. Experts in a specific field, like engineering, share a mutual cognitive environment: engineering. Similarly, a conference interpreter shares a mutual cognitive environment with his/ her audience, in form of the conference they are attending. Within this mutual cognitive environment, ``a piece of information will be taken as part of the intended context if it is the most accessible information that yields an adequately relevant interpretation'' \citep[2]{Gutt2004}. 


By definition translation brings people with different mutual environments together \citep[5]{Gutt2004}. As a secondary communication situation it lacks a mutual cognitive environment. Translator, source text author and reader do not share the same \isi{frame} of reference because they are separated in space and time. In the \isi{translation process}, the author's intentions have to be interpreted although they may not be explicitly stated in the text. These intentions have to be considered while rendering the text for the target public, a process for which it is also important to anticipate the target public's prior knowledge of the subject and the extent to which the author's aims and intentions consequently have to be adapted in order to be correctly communicated in the other language.


As second-order metacommunicative representations, translations should entail second-order metacognitive representations. A second-order metarepresentation is a metarepresentation standing for another metarepresentation. The first meta-representation, the source text, is already a higher-order representation since it stands for the author's ideas. The translator's primary concern is thus not the representation of a state of affairs, ``but the metarepresentation of bodies of thought'' \citep[13]{Gutt2004}. 


A metarepresentation is not a copy or duplication of a thought. Rather, translation is a transformation of metarepresentations. The source text is the only material basis for the generation of the translator's mental representation of the target text. The creation of the target text happens in the reverse order. It starts with the translator's purely mental representation of the author's mental representation as represented in the source text. During the \isi{translation process}, this mental metarepresentation is materialised in form of the target text. Translation briefs and technical guidelines offer indications both about author intentions and the background of the target audience. Such documentation is, however, not a default setting of translation. First, consider cases where this type of information is lacking, like in the case of dead authors. How does one translate ``a dead person, or a living person whom you never meet, or who never corresponds with you or your editor or your publisher in the attempt to control your work?'' \citep[24]{Robinson2001Who}. Second, the presence of extensive documentation does not exclude that they may be conflicting with the translator's views about author and audience. Finally, such documentation does not necessarily reduce the metarepresentational effort. To the contrary, it may even increase it since the translator has to add these ``external'' considerations about author and audience to the own assumptions of their respective cognitive environments. Instead of comparing the reference frameworks of two interlocutors and finding possible overlaps, the translator may end up juggling with the additional reference framework of the authors responsible for the technical documentation and translation brief. Given these possible complications, I shall treat the \isi{translation process} in what follows as a simple chain of \isi{text production} and re-production between author, translator and audience. Metacognition is a central feature of this process. 


Representing the minds of others is central to translation \citep{Wilss1992}. Traditionally, this feature of the translator's work has been studied in terms of imitation, empathy, metempsychosis and simulation. For Reiss and Vermeer, translation ``simulates a primary information offer'' (\citeyear[88]{Reiss1991}). For them, translation is an ``imitative action including the entire person'' (\citeyear{Reiss1991}: 91). Other translation scholars have pointed out that translation is ``inevitably mimetic'' \citep[249]{Mossop1998}, i.e. that it is always geared toward imitation \citep{Mossop1983, Mossop1998, Folkart1991, Gutt2000, Hermans2007}. Another key concept in this context is metempsychosis \citep[108]{Dussart1994}. This rather spiritual idea that one soul animates different bodies has been taken to explain the ``magnetism'' between translator and author \citep[241]{Wuilmart1990}. The term ``empathy'' has been used to describe the intuitive understanding between author and translator, a process that precedes rational understanding or goes beyond it \citep[109]{Dussart1994}. Folkart refers to recreation by translation as the ultimate form of mimetism (1991:418), and Stolze qualifies a full mimesis as the unreachable ideal of translation (\citeyear[144]{Stolze2010}). These concepts are seriously limited. Firstly, many of them cannot be used as parameters for empirical translation studies since they arise from traditional theoretical approaches to translation such as aesthetics. Furthermore, most of these concepts express rather general ideas. There is no translation-specific definition of ``imitation'' or ``empathy''. Different authors may use them to refer to different concepts. It is thus not very clear whether, in the context of Translation Studies, ``imitation'' and ``mimesis'' are to be thought of as distinct concepts. Similarly, it is unclear whether ``empathy'' and ``intuition'' cover the same phenomena and mechanisms.


Modelling the \isi{translation process} in terms of metarepresentations has at least three advantages. Metarepresentation provides a simple, yet powerful model of translation as a special form of inferential communication. It links up with previous research on metatexts and metacommunication, and accommodates them together with metacognition in a coherent framework. In this framework, translation is defined as a metacommunicative process generating metatexts. These texts are secondary communication situations about previous text. Generating them requires higher-order metarepresentation. These second-order metarepresentations should at least partially account for the \isi{cognitive effort} in translation. Furthermore, translators who are constantly operating on this higher metacognitive level should develop a higher cognitive proficiency than non-translators. I study these questions with the help of recent evidence from social psychology about \isi{attributive metacognition}, or Theory of Mind (ToM). 


\section{Theory of Mind}

Theory of Mind (ToM) describes the ability to represent and attribute mental states (such as beliefs, desires and intentions) to oneself and others \citep{Saxe2004}. It refers to ``our ability to reflect on ourselves and become self-conscious, and our ability to reflect on others and become conscious of the way others may see us. It involves thinking about how information is represented to us in terms of beliefs, desires and goals. Theory of mind is necessary for understanding the social world and our part in it'' \citep[31]{Larkin2010}. It allows us to make sense of others' behaviour and predict their future actions. Investigations into how the mind works have a long tradition in Western philosophy. For a long time, Theories about the Mind dealt with questions about how the mind could access itself -- as the only means to study one's thinking was to think about it. While there is a substantive body of research on how the mind deals with numbers, symbols and language, the research about the mental framework that deals with other minds is comparatively young. It is only in recent years that scientists with such diverse backgrounds like social psychology, neurosciences, anthropology and linguistics became interested in metacognition. Papers on neuroimagery research on Theory of Mind have increased from four in 2000 to more than 400 in 2013 \citep{KosterHale2013}.


The interest in mental state attribution began in \citeyear{Premack1978}, when Premack and Woodruff used the term ``theory of mind'' outside of philosophy to answer the question whether chimpanzees have a system of mental state attribution. They define the concept of theory of mind as the attribution of mental states to oneself and others. For their paper in particular, this idea applies beyond the boundaries of biological species. Their research consisted in presenting chimpanzees with a series of videotaped scenes showing human actors struggling with a variety of problems. After each video, the chimpanzee was presented with a picture featuring a possible solution to the problem. For example, one picture proposed a stick to reach for a banana which was too far away for the protagonist. The chimpanzees were consistently chose the photographs with the correct solution to the problem, which led the authors to infer that the animals were able to attribute a mental state to the actor (e.g. the desire to have the banana) and understand that they would regulate their behaviour according to their mental states. Although the study has received fundamental criticism \citep{Call2008}, it did not only spark the interest in the subject, but led the philosopher Daniel Dennett to think about other possible research designs for ToM testing \citeyearpar{Dennett1978}. Acknowledging that a fully-fledged Theory of mind was rather difficult to test, he asserts that the required conditions are easily met by communicative acts, such as warning, requesting or asking (ibid). While modern research is convinced that preverbal infants, apes and monkeys share any of the fundamental capacities of human social cognition, a fully-fledged Theory of Mind remains, like the sophisticated use of language, part of the uniquely human social cognition. 


The mature ToM network seems to be universal. Without any pre-existing neuroscience of ToM and unusually few preconceptions about its possible neural counterparts, every group that sought to identify brain regions implicated in ToM got essentially the same answer \citep{Saxe2010}. Activation in the same brain regions is found in participants ranging from 5 to 65 years of age from diverse regions of the world (Britain, USA, Japan, Germany, China, Netherlands and Italy) and in congenitally blind and deaf adults \citep{KosterHale2013}. To be so widely shared, neural substrates of ToM have to be similar in all these populations, and hence independent of the particular circumstances of their lives.


The regions reliably activated by ToM tasks are the right temporo-parietal junction (RTPJ) and the medial-prefrontal cortex (MPFC), the precuneus (PC) and the superior temporal sulcus (STS; \citealt{KosterHale2013, DodellFeder2011, Saxe2010, Young2010, Atique2010, Saxe2009}). All these brain regions have been identified through fMRI, transcranial magnetic stimulation (TMS) and lesion studies \citep{Saxe2009}.


But ToM is no default setting of the mind one is born with. It develops throughout childhood and undergoes significant changes until adolescence \citep{Gunther2012, Cummings2009}. 


Experience with diverse mental contents in language switch situations could help bilingual children to develop ToM competencies earlier than monolinguals \citep{Kovacs2009}. This argument is twofold: bilingual children do not only have experience with two languages, but also with mixing them both and switching from one language to another. A situation involving a language switch implies knowing that one of the communication partners does not understand one of the languages. Frequent exposure to such situations would lead to enhanced ToM capacities. Alternatively, the bilingual's experience with controlling multiple languages and adapt their use according to their environment could enhance the development of their executive control -- which in turn would enable them to perform better on ToM tasks that require such abilities \citep{Kovacs2009}. There is evidence that bilingual children know that and when interlocutors may not understand one of the child's languages (cf. \citealt{Bassnett2002}). Children learn to address their communication partners in the appropriate language before the age of three (ibid). Growing up with two languages confronts bilingual children more often with conflicting mental representations. A bilingual child has to learn that a monolingual friend does not understand what is being said in the child's second language. In a larger bilingual context, bilingual children have even been found mediating actively between two monolinguals by helping them by translating for them \citep{Kovacs2009}. Bilingual children grow up with multiple referents for objects. Whereas monolingual children only assign two labels to an object at around the age of 4, \isi{bilinguals} do so much earlier \citep{Kovacs2009}. Yet, there is no evidence suggesting that bilingual children may have advances language abilities. \citet{Kovacs2009} did not find any relation between the vocabulary scores and the ToM performance of bilingual children. Similarly, bilingual children perform better than their monolingual peers in false belief tasks and ToM tasks involving a language switch, without showing an advantage for either task. \citet{Kovacs2009} concludes that bilingualism enhances cross-domain performances. 


The influence of several languages may go beyond purely linguistic domains as bilingual individuals are at the same time bicultural. In an fMRI study by \citet{Kobayashi2008}, \ili{Japanese} (L1) and English (L2) bilingual children and adults were presented with false belief task in both languages. Whereas children's brain activation showed an overlap of activity for the L1 and L2 conditions, the brain activation patterns of adults varied depending on the task language. The results indicate that individuals recruit different neural resources depending on the language context, and that this difference may become greater with age. An alternative interpretation is that the different activation patterns are induced by the influence of participants' cultural background on their social cognition. Cultural influences can be found in terms of childrearing, mother-child interaction patterns and the way behaviour is explained to children \citep{Kobayashi2009}. 


Up to now, no systematic difference has been found in the development of ToM abilities depending on the child's mother tongue \citep[46]{Zufferey2010}. Similarly, bilingual and monolingual children achieve linguistic milestones at the same time. It is thus unlikely that a possible linguistic advantage alone could explain the superior performance of \isi{bilinguals} in ToM tasks. 


\Citet{VanOverwalle2009} proposes a comprehensive list of ToM tasks used in 200 fMRI studies, mainly published between January 2000 and April 2007. Among the tasks he identifies are: viewing tasks, tasks requiring imitation, a causal prediction or causality judgement. 


Non-verbal stimuli involve pictures of human faces, enacted human actions, comics and picture stories. Methodologies involving non-verbal stimuli include gaze tracking and non-verbal answers, e.g. by pushing buttons. In the so-called ``Mind in the Eyes'' test participants are presented with a series of 25 photographs of the eye-region of the face of different actresses and actors \citep{BaronCohen2001}. The picture is accompanied by four descriptive terms and participants have to select the one that offers the best description of the person's mental state. 


The most frequently used verbal stimuli are short stories and sentences. One of the earliest and often used test stimuli is the Strange Stories test by \citet{Happe1994}. The stories were originally designed as naturalistic tool for the diagnosis of specific ToM impairments in patients with autism. The original Strange Stories test consisted of 24 vignettes comprising 12 different types of stories with two stories for each type. The 12 different story types depict common elements of communication or communication situations, such as: lie, white lie, joke, pretend, misunderstanding, persuade, discrepancies between appearance and reality, figures of speech, sarcasm, forget, double bluff and contrary emotions. Adapted versions of the task contain stories on human mental and physical states as well as physical states of animals \citep{White2009}. FMRI item analyses reveal that activation of the ToM network does not depend on the linguistic features of the stimuli \citep{DodellFeder2011}. These findings suggest that the examined verbal ToM stimuli work independent of language -- and languages. 


Applying the ToM concept to translator's metarepresentation of other minds has several benefits. Unlike the previously mentioned concepts, ToM has been investigated thoroughly in numerous contexts, across several cultural and linguistic groups, covering all ages from early childhood to adulthood. ToM is associated with a robust activation pattern in fMRI studies, which is rare for comparatively complex cognitive phenomena. This activation pattern is independent of language. Furthermore, mother tongues do not seem to have an influence on ToM development.


While both mono- and bilingual children reach the different stages of their ToM development at the same time, bilingual children have shown to score better in ToM tasks than their monolingual peers. This difference is not necessarily explained by the fact that they speak different languages, but by how they use their languages. Children growing up in a bilingual environment frequently encounter situations in which they have to decide which type of verbal behaviour is most appropriate for their given audience: switching from one language to another because the audience does not share the same language; or mixing language because all parts of the audience share the same languages as the interlocutor, or translating what is said for the part of the audience that does not understand one of the languages used. Regular inferences on the content of other minds taking part in any given communication situation, and adapting one's behaviour to those inferences could help \isi{bilinguals} to acquire ToM more efficiently than their monolingual peers.


These observations make ToM a relevant concept for translation. Unlike the previously presented traditional concepts, ToM provides a model for empirical research. Linking existing research about the understanding of others in human communication in general with findings from Translation Studies will deepen our understanding of translation as a specialised form of human communication. Given that ToM is associated with a robust brain activation pattern, research about the role of ToM in translation could constitute one of the first steps into researching the neurological mechanisms of translation which are still one of the chief known unknowns of translation studies \citep[83]{Tymoczko2012}. The following section presents evidence for the role of \isi{attributive metacognition} in translation, and reasons why translation is likely to train this particular competence. 


\section{Theory of Mind in translation}

The translator's task does not tolerate any approximate use of language. Text creation based on prior text requires highly conscious choice of words, information structure and stylistic devices. The two main tasks of translation are reading and writing. Both activities have been shown to increase \isi{attributive metacognition}. Finally, translation training in classroom settings also leads to greater metacognitive competences.


Translating for other people is a formidable way to get experience with the way other speakers use their words and phrases. Translation involves a more conscious language use than direct monolingual standard communication when one speaks on behalf of oneself. ``I never realised what an imprecise word `clear' was until I tried to translate it'', says linguist Arika \citet[67]{Okrent2010} about her experience with translation. In spontaneous speech, we can use language without knowing what we want to say from the beginning; we can figure it out as we go along. Translators, in contrast, always have to know precisely what it is they are saying when they translate. 


Using several languages also entails a certain familiarity with different social conventions. Work by \citet{Shatz2006} shows that translators manipulate the expression of mental states in translation. For their study, Shatz and colleagues \citeyearpar{Shatz2006} developed a technique they call ``double translation'': two \isi{bilinguals} translate two versions of a book, the source text and a published translation. By comparing their work with the official translations, ``non-native researchers (\ldots) could note when the professional translators had translated something in a way that seemed unusual to them'' (2006: 96). Results of the study show that many modifications in the translations are motivated by culture-specific practices and beliefs. This shows that translators are sensitive to the culture-specific cues of mental states. In dealing with a particular mental state, translators reflect common beliefs or practices in a given culture. That is, they infer the mental state in question and adapt it to the social conventions of the target text culture. The increased demand in social reasoning imposed by the task is one of many factors influencing metacognitive abilities in translation.


Indirect communication trains perspective-taking capacities \citep{Djikic2013}. Among the best examples for the power of indirect communication are literature and narration. As soon as the reader starts engaging with the story, his/ her mind is almost automatically pulled out of his/ her actual present situation into the life of others. Reading means accessing this abstract, yet high concentration of social life. Reading means mind-reading.


Reading literary fiction has been found to improve ToM \citep{Djikic2014}. The more people read, the better they score on ToM tests \citep{Djikic2014}. Literature is the indirect experience of the other. Literature can be persuasive and lead the reader into an indirect communication with the characters. More generally, however, one of the main traits of literature is its subject matter. Literary writings deal with selves and their interactions in the social world \citep{Djikic2014}. The reader is taken to adopt the perspective of another and live, at least partially, through their experiences.


\largerpage
Translation and narration are based on the same principle: the willing suspension of disbelief \citep{Pym1998}. Although the translator is not the CEO of a bank, s/he will have to write the address to the reader in the annual report as if s/he was. In translation, both language comprehension and production require the willing suspension of disbelief which makes writing for translation yet another exercise in indirect communication. Individuals who have been writing fiction for several years scored higher on ToM tests \citep[17]{Djikic2013}. 


Translation involves a great share of reading, writing and hence confrontation with different types of higher language use, such as irony. Like narration, translation helps to understand language and the human mind as representational devices of cognition. Reading and writing are proven to train \isi{attributive metacognition}. Reading also increases evaluative metacognitive skills such as \isi{self-monitoring} and control skills \citep[74]{Larkin2010}. 


Translation comes intuitively to mind as one of the best ways to engage people in perspective-taking. The psychologists Emile Bruneau and Rebecca Saxe \citeyear{Bruneau2012} tested members of two conflict groups in a perspective-giving and perspective-taking paradigm. Two roles were assigned to participants of each group, Sender and Responder. In the perspective-giving task, participants with a Sender-role had to write a brief description about difficulties and challenges of their respective situation. Responders were told that the brief was a translation they had to verify. They had to summarize the Sender's statement in their own words, but without expressing their own beliefs, feelings and experiences. According to the authors, ``describing the difficulties and challenges experienced by the outgroup in one's own words is a novel and robust implementation of perspective-taking'', since it requires the Responder ``to at least partially get `inside' the Sender's description'' \citep[856]{Bruneau2012}.


Classroom experience with translation also trains perspective-taking capacities. \citet{Salles2010} analysed whether translation activities help second language learners to become aware of the L2 perspective and consequently adjust to it to improve their ability to effectively communicate in the foreign language. The ability to communicate effectively in a second language is highly dependent on the ability to conform to the perspective of the second language (2010: 1), as reflected in deictic elements. Likewise, L2 learners have to be aware of possible influences certain grammatical structures have on the mental representation of text (e.g. passive voice as compared to active voice). Salles Rocha points out that ``professional translators do not only have to be aware of the different perspectives embodied in the language that they are dealing with, but also know how to take those perspectives when passing from one language into another'' (2010: 8). In her study, she compares the organisation of information in descriptive essay by native and non-native speakers, i.e. American undergraduate students and mostly \ili{Chinese} English-language learners, before and after translation exercises. After the translation exercises the learners got closer to the way native speakers conveyed information, improving a significant number of thematic and processual structures \citep[44]{Salles2010}, in particular their use of material and mental state processes (2010: 45). The findings indicate the learners gained greater awareness of how native speakers compact and organise information (2010: 45).

In 1989, Miriam Shlesinger launched the two year Translation Skills Program (TSP) for some secondary schools in Israel. It proposes classes in which students translate from English (L2) intro Hebrew (L1). In a longitudinal study, \citet{Shlesinger2011} investigated the effects of the TSP on students' metalinguistic awareness. As a result oft he study the authors consider \isi{translation competence} as the interplay between metalinguistic awareness and general language skills: ``Translational proficiency might be thought of as an interplay between bilingual proficiency and meta-linguistic maturity, involving the recognition of commonalities and differences in the nature and functions of languages, analysis of linguistic knowledge and control over processing'' \citep[164]{Shlesinger2011}.

Translation scholars predict and observe similar changes in the academic translation classroom. Dam-Jensen \& Heine point out the importance of the text producer's mental state and its interaction with the ``situation in which it evolves'' (\citeyear[91]{DamJensen2013}). Author and audience influence this situation. \citet{Shreve2009} emphasised the role of the translator's position regarding these two, suggesting a shift during the development of translation expertise. According to him, translation experts focus on the target audience, whereas \isi{novices}' attention would solely lie on the source text. Research by Ehrensberger-Dow \& Massey provides evidence for these predictions. Translation \isi{novices} in their study use comments on the readership to solely refer tot he ST readers. In line with \citet{Shreve2009}, a more equilibrated view regarding the implication of others in translation reflects the emerging awareness of the translator's position and the multiple roles s/he has to handle. Again, data by \citet{EhrensbergerDow2013} confirms this prediction. MA students were found to spread their attention ``over three categories, with half of them indicating an awareness of the importance of conveying the message of the ST and tending to talk about target text readerships'' (2013: 111). 

In summary, this section provided evidence to support the hypothesis that translation requires ToM. The translator represents both source and target other at the same time. Forming these concurrent metarepresentations should hence activate the ToM network in the brain. Furthermore, frequent exposure to translation should train ToM.

To test this hypotheses, I compare participants with two different levels of \isi{translation competence}: BA and MA students from the Translation Faculty at the University of Geneva. In the present framework,  BA students considered non-proficient, or novice translators. MA students assumed to have a greater \isi{translation competence} due to more training. If it is true that translation involves ToM, a translation task should engage the neural ToM network. However, brain activation patterns should be different for BA and MA students if it is true that translation trains ToM. That is, I assume different levels of \isi{translation competence} to be associated with different activation patterns of the ToM network. Details of the study are presented below.

\section{Study}

Subjects were presented with 40 \ili{German} sentences, 20 of which were in a ToM condition and 20 in a noToM condition. The task consisted in reformulated each sentence in the same language. This was to avoid \isi{noise} in the neuroimagery data due to participants' different language levels. Two sentences of each condition were matched in terms of sentence \isi{construction} as to exclude effects due to linguistic particularities or simple lexical processing. A ToM condition sentence requires participants to take the narrator's perspective in order to infer the meaning of the message (e.g. ``When I stood on the stage for the very first time, my palms became wet''). For the noToM condition sentences, the simple understanding of the sentence's logic was required (e.g. ``When touching that used towel, my palms became wet''). The resemblance of the sentences should guarantee that there was no effect linked to any text-statistics factor that would influence the results \citep{DodellFeder2011}.

24 subjects (13 BA, 11 MA) were tested. Functional data were collected on 3T-MRI scanner (Siemens), analyzed with SPM8 using fixed-effect analysis with a general linear model applied to each voxel and an auto- regressive function to account for temporal correlations between them across the whole brain. Afterwards, simple main effects of each condition were subjected to a random-effect analysis. All conditions were modeled in a full 2x2 factorial model (ANOVA) with modalities (verbal/ nonverbal) as factor 1 and the condition (ToM/ noToM) as factor 2.

Participants were asked provide an intralingual translation of the sentence, focusing on the sentence's message. The intralingual translation setting was chosen because it allows for a better control of the design, particularly with respect to possible influences different degrees of language proficiency could have on the brain activation patterns \citep{Kim1997, Korning2009}. The baseline task of the verbal condition consisted in reading aloud. As non verbal control condition, the Mind in the Eyes task was chosen. This test was originally developed as a diagnostics tool for autism by \citet{BaronCohen1997}. It consists of a set of pictures showing only the eyes of a person. In the ToM condition, participants have to choose one adjective out of four to describe the expression of the eyes. The original task was complemented by a no ToM condition consisting in attributing an age to the depicted person, and a baseline task in which participants have to indicate the location of a red dot placed in the picture.

\section{Results and discussion}
\largerpage
The ToM-noToM contrast for the verbal task reveals an important activation in the left middle temporal gyrus, the left precuneus, the left cerebellum, the left middle inferior temporal gyrus, the left middle temporal gyrus, the left caudate body, the left subgyral part of the left temporal lobe, the left parahippocampal gyrus and the right superior temporal gyrus. The interaction analysis for the ToM-noToM contrast reveals activation in four regions across the non-verbal and verbal modality: the bilateral precuneus, the right superior frontal gyrus the inferior temporal gyrus and the left cerebellum.

The hypothesis that translation activates the ToM network can thus be only partially confirmed since the only ToM area activated by the task is the precuneus (PC). Despite being an important part of the ToM network, the literature dedicated to the role of the precuneus is scarce, but it is a major association area and is implied in numerous behavioural functions, such as visuo-spatial imagery, episodic memory retrieval, self-processing and consciousness \citep{Cavanna2006}. Its implication in self-processing seems to be relevant to my study, because first-person reference (`I', `my', `me') was used in all verbal stimuli. Could the activation of the PC be due to the participants' processing of this self-reference rather than to the translation condition? Experiments revealing the implication of the precuneus in self-processing have addressed with the representation and awareness of the self \citep{Cavanna2006}, more precisely with the representation of self versus non-self representation as in self-referential judgement and first- versus third-person-perspective-taking. These studies involved tasks in which participants were asked to compare self-relevant traits with self-irrelevant traits of information \citep{Cavanna2006} by asking them to make decisions about psychological personality trait adjectives, or attributing personality trait adjectives to themselves \citep{Cavanna2006}. Other studies found activation in the PC when participants were asked to describe themselves as compared to a neutral reference person \citep{Cavanna2006}. Further evidence for the activation of the PC was found in studies asking the participants to evaluate psychological traits they associated with three people with different degrees of self-relevance, namely the person herself, her best friend and a neutral person \citep{Cavanna2006}.

The studies listed above provide evidence for the fact that activation in the PC is linked to various forms and degrees of self-reference. However, self-reference cannot fully account for the PC activation in my study because the nonverbal task does also requires the first-person-vantage point. Although the latter might be rather implicit in the nonverbal condition, the task consists of the evaluation of others' facial expression as seen by me. The choice of answers could thus be rephrased as `To me, he looks aggressive', or `I think she looks flirting'. First-person agency does thus not depend of the test condition. However, this factor might be more explicit in the verbal condition, because the personal and demonstrative pronouns `I', `my' and `me' might trigger a more explicit form of self-reference than the non verbal condition. The factor of self-reference might therefore be stronger in the verbal condition as compared to the nonverbal condition.

This view finds further support by an fMRI item-wise analysis of theory of mind tasks which revealed that the number of people in a story was the best predictor for activity in the PC. Activation was greater if more people were involved \citep{DodellFeder2011}. According to this meta-analysis, the nonverbal task in my study should have elicited a greater PC activation because it involves several different people as compared to the translation condition, which features only one protagonist, the `I'. Self-reference cannot explain the activation patterns found in the present study.

However, PC activation in studies about intergroup conflict seems to be more elusive with regards to the translation task in my study. \citet{Bruneau2010} found that the activity in the precuneus was strongly correlated with explicit and implicit behavioural measures of negative attitudes towards the outgroup. They presented Arab and Israeli participants with statements of partisan views and measured the BOLD response with fMRI imaging. Only the PC was reliably recruited during emotion- laden reasoning in most individual subjects. Furthermore, only the PC differentiated between pro- ingroup and pro-outgroup statements across groups. \citet{Bruneau2010} provide one of the rare neuroimagery studies where implicit associations towards the outgroup have been studied. Their study is in line with other work about the PC's implication in emotional reasoning. PC activation has been reported when participants with a very strong political orientation were confronted with apparent contradictory statements made by their own political candidate \citet{Bruneau2010}.

\largerpage%longdistance
Similarly, the verbal stimuli used in my study most likely did not reflect the participants' own view. The PC activation revealed in the verbal task might thus reflect the inner conflict of using terms of self- reference such as `I', `my' and `me' without, however, actually referring to oneself \citep{Robinson2001Who, Hermans2007, Pym2005The}. While using these words, the translator, very clearly, distinguishes herself from the person whose place she takes while producing these utterances. The PC activation might thus reflect this inner disparity between intended reference, i.e., the author, and the actual performing reference, i.e., the translator.

The hypothesis that different levels of \isi{translation competence} would be related to different levels of activation of the ToM network could not be confirmed. There are three possible explanations to account for this result:

\begin{enumerate}
\item There is no difference between both groups.
\item There are differences between the participants, but the actual group distinction is not a sensitive criterion for them.
\item There are differences between both groups, but the verbal test is not sensitive enough to detect them.
\end{enumerate}

The first point can only be reliably addressed by further testing in terms of neuroimagery studies and in terms of other experimental research into the metacognitive proficiency of translators. The second possible explanation could be that the verbal task is not sensitive enough to reveal group differences for the studied conditions. The verbal task did not yield any group differences in terms of brain activation. Similarly, the nonverbal task did not reveal any group difference in terms of reaction time. Both tasks, the nonverbal and the verbal, yielded a robust contrast for both conditions (ToM and noToM). The third point could be due to a ceiling effect in the brain activation. Because all subjects were healthy young students of a similar age range, the distinction of academic curricula might not be sensitive with regards to differences in the activation of neural networks.

However, BA students showed a greater activation in the inferior parietal lobule throughout the verbal task. This activation was, however, independent of condition.
The intralingual translation task parallels the results of the validated nonverbal test design, and therefore seems to be adequate for the testing of conditions. The most plausible explanation may thus be that the group distinction is not a sensitive criterion to answer the question. It has been observed before that professional translators do not necessarily produce high quality translations \citep{Sun2014}. 

Since the BA/MA distinction may not be sensitive enough a criterion to detect actual differences in \isi{translation competence}, a third analysis was conducted for which participants were regrouped according to the quality ratings of their translations. This analysis revealed a positive correlation between \isi{translation quality} ratings and activity in the precuneus. In other words: greater precuneus activity lead to translations which received higher ratings for their quality. This finding emphasizes the previously discussed role of the precuneus. 

\section{Conclusion}
\largerpage
This paper argues that translation requires the metarepresentation of at least two other mindsets. If this assumption is right, translation should be highly demanding in terms of \isi{attributive metacognition}; and the latter should account at least for some of the \isi{cognitive load} involved in the translator's task. A second assumption is that frequent exposure to this task trains metacognitive abilities, among them \isi{attributive metacognition}, or ToM. 

I set up an fMRI study to answer the question whether translation activates the ToM network. This was the case. However, translation did not activate the entire ToM network, but only one part of it, the precuneus (PC). This area is most frequently associated with self-reference, self-processing and awareness of self. In addition, PC activation has been positively correlated to the number of people in a story. In the present context, it may indicate the translator's awareness for the multiple metarepresentations. ToM stimuli required the representation of several minds whereas noToM stimuli descriptively reported facts or states of the world. 

Furthermore, the results of the present study indicate a positive correlation between PC activation and \isi{translation quality} ratings. Higher PC activation was linked to higher \isi{translation quality}. This finding could indicate that successful \isi{attributive metacognition} contributes to \isi{translation quality}. 

However, this partial activation of the ToM network does not mean that subjects are consciously aware of textual requirements in terms of attributive meta\-cognition. Existing literature suggests that students' pragmatic awareness builds up slowly and is only acquired over time. Since \isi{attributive metacognition} is only one part of this pragmatic ability, its development may be even less visible. The training of metacognitive abilities could also evolve through other translation-related tasks, like reading or writing fiction. The training of other subcompetences, such as executive function, could also lead to better metacognitive abilities. 

On the methodological level, this study attempts to push disciplinary boundaries by studying a traditional translation concept like ``otherness'' with the help of neuroimagery. The present work shows that neuroimagery research need not be limited to localising translation in the brain. Transdisciplinary research in translation does not only further our understanding about translation, but also our understanding of what it means to be multilingual. 

\sloppy
\printbibliography[heading=subbibliography,notkeyword=this]
\end{document}

% \textbf{Bibliography}
% 
% Atique, Mohammed Naushad \citet{Bijoy2010}: Neuronale Grundlagen sozialer Kognition -- fMRT Untersuchungen zur Theory of Mind. PhD dissertation. Eberhard Karls Universität zu Tübingen.
% 
% Baron-Cohen, Simon; Wheelwright, Sally; Hill, Jacqueline; Raste, Yogini \& Ian \citet{Plumb2001}: The ``Reading the Mind in the Eyes'' Test Revised Version: A study with normal adults, and adults with Apserger Syndrome or high-functioning Autism. In: Journal of Child Psychology and Psychiatry, Vol. 42, n°2, pp. 241-251.
% 
% Baron-Cohen, Simon; Wheelwright, Sally \& Therese \citet{Jolliffe1997}: Is there a `Language of the Eyes'? Evidence form normal adults, and adults with Autism or Asperger syndrome. In: Visual cognition, Vol. 4, n°3, pp. 311-331.
% 
% Bassnett, \citet{Susan2002}: Translation Studies. 3\textsuperscript{rd} edition. Routledge. London \& New York.
% 
% Bigelow, \citet{John1978}: Semantics of Thinking, Speaking and Translation. In: Meaning and Translation. Philosophical and Linguistic Approaches. F. Guenthner and M. Guenthner-Reutter (eds.) Duckwort \& Company Ltd.p. 109-135.
% 
% Bruneau, Emile G.; Dufour, Nicolas \& Rebecca \citet{Saxe2012}: Social cognition in members of conflict groups: behavioural and neural responses in Arabs, Israelis and South Americans to each other's misfortunes. In: Philosophical Transactions of the Royal Society, Biological Sciences, Vol. 367, pp. 717-730.
% 
% Bruneau, Emile G. \& Rebecca \citet{Saxe2010}: Attitudes towards the outgroup are predicted by activity in the precuneus in Arabs and Israelis. In: NeuroImage 52, pp. 1704-1711.
% 
% Call, Josep \& Michael \citet{Tomasello2008}: Does the chimpanzee have a theory of mind? 30 years later. In: Trends in Cognitive Science, Vol. 12, n°5, pp. 187-192.
% 
% Cavanna, Andrea E. \& Michael R. \citet{Trimble2006}: The precuneus: a review of ist functional anatomy and behavioural correlates. In: Brain, Vol. 129, pp. 564-583.
% 
% Cummings, \citet{Louise2009}: Clinical Pragmatics. Cambridge University Press. Cambridge.
% 
% Dam-Jensen, Helle & Carmen Heine (2013): Writing and translation \isi{process research}: Bridging the gap. In: Journal of Writing Research, Vol. 5, n°1, pp. 89-101.
%
%
% Davidson, \citet{Donald2010}: Inquiries into Truth and Interpretation. Oxford University Press. Oxford.
% 
% Dennett, \citet{Daniel1978}: Beliefs about beliefs. In: The Behavioural and Brain Sciences, Vol. 4, pp. 568- 570.
% 
% Djikic, Maja \& Keith \citet{Oatley2014}: The Art in Fiction: From indirect communication to changes of the Self. In: Psychology of Aesthetics, Creativity, and the Arts, Vol. 8, n°4, pp. 498-505.
% 
% Djikic, Maja; Oatley, Keith and Mihnea C. \citet{Moldoveanu2013}: Reading other minds. Effects of literature on empathy. In: Scientific Study of Literature, Vol. 3, n°1, pp. 28-47.
% 
% Dodell-Feder, David; Koster-Hale, Jorie; Bedny, Marina; Saxe, \citet{Rebecca2011}: fMRI item analysis in a theory of mind task. In: Neuroimage, Vol. 55, n°2, pp. 705-712.
% 
% Dussart, André (1994): L'empathie, esquisse d'une théorie de la reception en traduction. In: Meta: journal des traducteurs/ Meta: Translators' Journal, Vol. 39, n°1, pp. 107-115.
% 
% Ehrensberger-Dow, Maureen \& Gary \citet{Massey2013}: Indicators of \isi{translation competence}: Translators' self-concepts and the translation of titles. In: Journal of Writing Research, Vol. 5, n°3, pp.103-131.
% 
% Fletcher, P.C.; Frith, C.D; Baker, S.C.; Shallice, T.; Frackowiak, R.S.J.; Dolan, R.J. (1995): The Mind’s Eye – Precuneus Activation in Memory-Related Imagery. In: Neuroimage, n°2, pp. 195-200.
%
%
% Folkart, \citet{Barbara1991}: Le conflit des énonciations. Editions Balzac. Candiac. Québec.
% 
% Gunther Moor, Bregtje; Op de Macks, Zdena A.; Güroglu, Berna; Rombouts, Serge A. R. B.; Van der Molen, Maurits; Crone, Eveline A. (2012): Neurodevelopmental changes of reading the mind in the eyes. In: SCAN, no°7, pp. 44-52.
% 
% Gutt, Ernst-\citet{August2004}: Challenges of Metarepresentation to Translation Competence. In: Fleischmann, E.; Schmitt, P.A.; Wotjak, G. (eds.): Translationskompetenz. Tagungsberichte der LICTRA. Tübinge. Stauffenberg, pp. 77-89.
% 
% Gutt, Ernst-\citet{August2000}: Translation and Relevance. Cognition and Context. St Jerome Publishing. Manchester, Kinderhook (NY) USA.
% 
% Happé, \citet{Francesca1994}: An advanced test of theory of mind: Understanding story characters' thoughts and feelings by able autistic, mentally handicapped, and normal children and adults. In: Journal of autism and developmental disorders, Vol.24, n°2, pp. 129-154.
% 
% Hermans, \citet{Theo2007}: The conference of the tongues. St Jerome Publishing. Manchester, Kinderhook (NY) USA.
% 
% Kelly, Nataly \& Zetzsche, \citet{Jost2012}: Found in Translation. How Language Shapes Our Lives and Transforms the World. Penguin Group Inc. New York, USA.
% 
% Kim, Karl H. S.; Relkin, Norman; Lee, Kyoung-Min, Hirsch, \citet{Joy1997}: Distinct cortical areas associated with native and second languages. In: Nature, Vol. 388, pp. 171-174.
% 
% Kobayashi Frank, Chiyoko \& Elise \citet{Temple2009}: Cultural effects on the neural basis of theory of mind. Chapter 15, in: Progress in Brain Research, Vol. 178, pp. 213-223.
% 
% Kobayashi, Chiyoko; Glover, Gary H.; Temple, \citet{Elise2008}: Switching language switches mind: linguistic effects on developmental neural bases of `Theory of Mind'. In: Social Cognitive and Affective Neuroscience, Vol. 3, n°1, pp. 62-70.
% 
% Korning Zethsen, \citet{Karen2009}: Intralingual Translation: An Attempt at Description. In: Meta: journal des traducteurs/ Meta: Translators' Journal,Vol. 54, n°4, pp. 795-812.
% 
% Koster-Hale, Jorie \& Rebecca \citet{Saxe2013}: Functional neuroimaging of theory of mind. In: Simon Baron-Cohen, Helen Tager-Flusberg, Michael V. Lombardo (eds.): Understanding other minds. Perspectives from developmental social neuroscience. Oxford University Press. Oxford, pp. 132-164.
% 
% Kovacs, Agnes \citet{Melinda2009}: Early bilingualism enhances mechanisms of false-belief reasoning. In : Developmental Science, Vol. 12, n°1, pp. 48-54.
% 
% Larkin, \citet{Shirley2010}: Metacognition in young children. Routledge. Abingdon (Oxon) \& New York.
% 
% Mossop, \citet{Brian1998}: What is a translating translator doing? In: Target, Vol. 10, n°2, pp. 231-266.
% 
% Mossop, \citet{Brian1983}: The Translator as Rapporteur: a Concept for Training and Self-Improvement. Meta Vol. 28, pp. 244-278.
% 
% Okrent, \citet{Arika2010}: In the land of invented languages. Spiegel \& Grau. New York.
% 
% Plassard, \citet{Freddie2007}: Lire pour traduire. Presses de Sorbonne Nouvelle. Paris.
% 
% Popovic, \citet{Anton1976}: Aspects of metatext. In: Canadian Review of comparative literature/ Revue canadienne de literature comparée. Fall/\citealt{Automne1976}.
% 
% Premack, David \& Guy \citet{Woodruff1978}: Does the chimpanzee have a theory of mind? In: The Behavioural and Brain Sciences, Vol. 4, pp. 515-526.
% 
% Pym, Anthony (1998): Method in Translation History. St. Jerome Publishing, Manchester.
%
%
% Pym, \citet{Anthony2005}: The translator as non-author, and I am sorry about that. In: Claudia Buffagni, Beatrice Garzello \& Serenella Zanotti (eds.): The translator as author. Perspectives on Literary Translation. Proceedings of the International Conference. Lit Verlag. Berlin, pp. 31-43.
% 
% Reiss, Katharina & Vermeer, Hans-J. (1991): Grundlegung einer allgemeinen Translationstheorie. 2. Auflage. Niemeyer Verlag, Tübingen.
%
%
% Robinson, \citet{Douglas2001}: Who translates? Translator subjectivity beyond reason. State University of New York Press. New York, USA.
% 
% Salles Rocha, \citet{Josiany2010}: Translation and perspective-taking in the second-language classroom. MA thesis. Kent State University.
% 
% Saxe, \citet{Rebecca2010}: The right temporo-parietal junction: a specific brain region for thinking about thoughts. In: Alan Leslie \& Tamsin \ili{German} (eds.): Handbook of Theory of Mind. Psychology Press. Hove.
% 
% Saxe, \citet{Rebecca2009}: Theory of Mind (Neural Basis). In: William P. Banks (ed.): Encyclopedia of Consciousness. Academic Press. London, Oxford, Boston, New York, San Diego, pp. 401-409.
% 
% Saxe, Rebecca; Carey, Susan; Kanwisher, \citet{Nancy2004}: Understanding other minds : Linking developmental psychology and functional neuroimaging. In: Annual review of Psychology, Vol. 55, pp. 87-124.
% 
% Shatz, Marilyn; Dyer, Jennifer; Marchetti, Antonella; Massaro, \citet{Davide2006}: Culture and mental states: A comparison of English and \ili{Italian} Versions of Children's Books. In: Antonietti, Alessandro; Liverta-Sempio, Olga; Marchetti, Antonella (Eds.): Theory of Mind and Language in Developmental Contexts. Springer. New York.
% 
% Shlesinger, Miriam and Ruth \citet{Almog2011}: A new pair of glasses. Translation skills in secondary school. In: Cecilia Alvstad, Adelina Hild and Elisabet Tiselius (eds.): Methods and Strategies of Process Research. John Benjamins. Amsterdam, Philadelphia, pp. 149-169.
% 
% Shreve, Gregory \citet{Monroe2009}: Recipient-orientation and metacognition in the \isi{translation process}. In: Rodica Dimitriu \& Miriam Shlesinger (eds.): Translators and Their Readers: In Homage to Eugene A. Nida. Les Editions du Hasard, Brussels, pp. 257-270.
% 
% Stolze, \citet{Radegundis2010}: «Hermeneutics and Translation» In: Yves Gambier \& Luc van Doorslaer (eds.): Handbook of Translation Studies. Vol. 1. John Benjamins. Amsterdam, Philadelphia.
% 
% Sun, Sanjun \& Gregory M. \citet{Shreve2014}: Measuring translation difficulty: An empirical study. In: Target, Vol. 26, n°1, pp. 98-127.
% 
% Tymoczko, \citet{Maria2012}: The neuroscience of translation. In: Target, Vol. 24, n°1; pp. 83-102.
% 
% Van Overwalle, \citet{Frank2009}: Social cognition and the brain: a meta-analysis. In: Human Brain Mapping, Vol. 30, n°3, pp. 829-858. 
% 
% White, Sarah; Happé, Francesca; Hill, Elisabeth; Frith, \citet{Uta2009}: Revisiting the Strange Stories: revealing mentalizing impairments in autism. In: Child Development, Vol. 80, n°4, pp. 1097-117.
% 
% Wilss, \citet{Wolfram1992}: Übersetzungsfertigkeit. Annäherungen an einen komplexen übersetzungspraktischen Begriff. Gunter Narr Verlag. Tübingen.
% 
% Wuilmart, Françoise (1990): Le traducteur littéraire : un marieur empathique de cultures » Meta : journal des traducteurs/ Meta: Translator's Journal, Vol. 35, n°1, pp.236-242.
% 
% Young, Liane; Camprodon, Joan Albert; Hauser, Marc; Pascual-Leone, Alvaro \& Rebecca \citet{Saxe2010}: Disruption of the right temporoparietal junction with transcranial magnetic stimulation reduces the role of beliefs in moral judgements. In: PNAS, Vol. 107, n°15, pp. 6753-6758.
% 
% Zufferey, \citet{Sandrine2010}: Lexical pragmatics and Theory of Mind. John Benjamins. Amsterdam/ Philadelphia.
