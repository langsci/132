\documentclass[output=paper]{LSP/langsci} 
\ChapterDOI{10.5281/zenodo.1090994}
\title{Cognitive economy and mental worlds: Accounting for translation mistakes and other communication errors}
\author{Pertti Hietaranta
\affiliation{University of Helsinki}    
}
% \textit{pertti.hietaranta@helsinki.fi} 

\abstract{The present paper applies the two notions of cognitive economy and mental world to the analysis of two rather different cases of miscommunication in translation. The paper argues for two tenets. First, the paper argues that cognitive economy is occasionally manifested as an unwarranted and only partly conscious decision to switch over, in the construction of translations, to what \citet{Berger2007} calls the experiential mode of information processing. This is a way of processing information which is not analytical in nature but rather based on intuition generated by past experience and thus susceptible to overlooking some crucial source text information. This is information which is essential to the construction of adequate translations and which can be detected by the brain in its more methodically oriented rational mode only. Secondly, the paper also argues that the notion of mental world can be invoked to account for certain aspects of cultural infelicities in translations.


A minor part of the material discussed in this paper was presented in a preliminary form at the 6\textsuperscript{th} DGKL conference Constructions \& Cognition (Erlangen, Germany, Sept 30--Oct 2, 2014) but the gist of the argumentation, especially the part related to the conclusions reached, is based on my later contribution to the Translation in Transition II conference (Germersheim, Germany, Jan 29--30, 2015).}
\rohead{\thechapter\hspace{0.5em}Cognitive economy and mental worlds}
\maketitle
\begin{document}

\section{Introduction: Errare humanum est}\label{hietaranta:sec:1}

We know from experience that we occasionally make the wrong choice when there are more courses of action than one available in the situation we happen to be in. This is a phenomenon which makes itself known in linguistic as well as in non-linguistic contexts.


Trying to find the shortest way back to your hotel in a city you are not very well acquainted with is a familiar example of the non-linguistic variety. A few minutes after you have started walking in what you believe is the right direction you realise you did not take the shortest route after all but rather took the wrong turn at some point and ended up taking, inadvertently, a more or less distinctly longer detour.


Given the existence of such experiences, it is quite interesting to note that a recent paper by \citet{Holscher2011}, which explicitly discusses the ``communicative and cognitive factors influencing planning strategies in the everyday task of choosing a route to a familiar location\ldots'' (228), is a study which concludes that ``different planning and navigation conditions lead to different wayfinding strategies'' (245). Given this result, it is hardly surprising that there may be different wayfinding strategies employed also in planning routes to \textit{unfamiliar} locations, and that some of those strategies lead to (navigation) results which do not fully meet the planner's expectations but rather lead one astray.


Misunderstanding a text in turn exemplifies the linguistic type of an incorrect choice: mistranslations occasionally occur because a translator interprets the source text in a way which is in some respect(s) different from the intentions of the author of the original text and thus ends up producing a translation which is considered to contain one or more mistakes due to such a misunderstanding of the original author's intention(s). Arguments similarly sometimes surface because one person interprets another person's words in an unexpected manner, resulting in the all too familiar ``that's not what I meant'' conversation.


The present paper takes a detailed look at two attested cases where the wrong choice is made when a text is received, interpreted, and ultimately understood (in a manner not intended by the sender), and seeks to offer a cognition-based explanation as to why a linguistic \isi{construction} is sometimes misunderstood when there are two equally sensible readings available, i.e. why language users occasionally make the wrong choice when selecting a given semantic (and pragmatic and functional) interpretation as the basis of their understanding of the text.
 
Translation is an activity which is crucially based on constructions. We do not translate texts word by word, not in professional settings anyway, but rather by larger chunks, by constructions and combinations of constructions. Therefore, if something goes wrong with a particular translation assignment, we would probably do wisely to start looking for the cause(s) of the failure by checking first if the analysis of the source text made by the translator is one where the constructions constituting the entire text is of the kind that makes it plausible to argue that the sum total of the constructions in the translation indeed accounts for the contents and structure of the entire original text. Further, we wish to make sure that, between the source text and the target text, i.e. the translation, there is adequate correspondence (or equivalence or matching in the sense of \citealt{Holmes1988}), i.e. correspondence which makes the source text and the translation similar to each other in an appropriate manner and to an appropriate extent so that we can justifiably call the latter a translation of the former rather than a version of the former. Yet, it sometimes happens that these requirements are not fully met and that the desired goal is consequently not reached after all.

 
  As human beings, we are different from machines in a number of respects, but for our present concerns there is one particular difference which is of considerable significance: while even the most sophisticated machines (or computer programs, which may be viewed as a special type of machine, cf. e.g. \citealt{rammert2008}) can only be made to make observations on the world, human beings can go further, viz. we can make sense of the world, that is, we can interpret and thereby understand the world and give meanings to its various phenomena. This in turn is crucially based on the fact that we continuously make assessments of and judgments on what we observe around us, the results of these assessment or judgment operations then making us do yet other things: most notably, we make decisions based on our assessments in order to reach the goals we find desirable.

  In support of this view of human behaviour we can note, for instance, that there is a paper by \citet[615]{Lupyan2013} which argues convincingly that ``educated adults routinely make errors in placing stimuli into familiar, well-defined categories\ldots'', and that ``the distributed and graded nature of mental representations means that human algorithms, unlike conventional computer algorithms, only approximate rule-based classification and never fully abstract from the specifics of the input''. That is, since we are not all that good at context-free computation, we occasionally interpret a situation inadequately and make the wrong choice: ``If human algorithms cannot be trusted to produce unfuzzy representations of odd numbers, triangles, and grandmothers, the idea that they can be trusted to do the heavy lifting of moment-to-moment cognition that is inherent in the \isi{metaphor} of mind as digital computer still common in cognitive science needs to be seriously reconsidered'' (ibid.).

   
  Here, translation is no different from other forms of human action which require decision making based on analyses where assessments, judgments or evaluations of data play a major role. Language, however, does present some complexities of its own. If we agree with \citet[457--458]{Langacker2008} that ``a discourse comprises a series of usage events\ldots'' and that ``conceptually, a usage event includes the expression's full contextual understanding -- not only what is said [or written] explicitly, but also what is inferred, as well as everything evoked as the basis for its apprehension,'' it becomes understandable that we occasionally miss something in such a complex project and end up generating, in translation, another series of usage events in another language which is not quite adequate for the purpose at hand and that misunderstandings accordingly sometimes occur. After all, it is pieces of language in discourse that we translate: cf. e.g. Halverson's (\citeyear{Halverson2013}: 34) remark to the effect that ``the creation of translation\ldots, in whatever medium, is recognized by translation scholars as an instance of discourse; that is, as a communicative event situated in historical, cultural, and personal circumstances and impacted by the particulars of those very real circumstances''.

  
  This view seems to be compatible, in all essential respects, with what \citet[98]{Croft2004} say about meanings, in particular when they argue that ``words do not really have meanings, nor do sentences have meanings: meanings are something that we construe, using the properties of linguistic elements as partial clues, alongside non-linguistic knowledge, information available from context, knowledge and conjectures regarding the state of mind of hearers and so on''. On such a conception of language in general and of meanings in particular, it is understandable that language users should make mistakes of especially the misunderstanding variety: if meanings are not fixed products of reification or of other similar processes but are rather the inferred end results of dynamic cognitive processes, it is more than likely that language users sometimes pay less attention than is desirable or required to what they hear or read and for that reason fail to arrive at the interpretation their interlocutor had in mind when sending the message. Therefore, as long as a person's state of mind has an effect on how fully or appropriately the person's brain can process the incoming message, it is understandable that mistakes are occasionally made: if a person is distracted from processing the events of incoming discourse, there may not be enough processing or monitoring capacity available to pick out all of the salient clues from the spoken or written text the person is receiving, in which case the inferential processes applied to the language data received and processed may generate a misunderstanding.\footnote{That judgment and decision making are, essentially, complex cognitive tasks affected by a variety of conditions is made exceptionally clear by \citet[54]{Weber2009}, who strongly argue for the tenet that, although earlier studies of judgment and decision making were ``dominated by mathematical functional relationship models, ... the field has started to realize, however, that the brain that decides how to invest pension money and what car to buy is the same brain that also learns to recognize and categorize sounds and faces, resolves perceptual conflicts, acquires motor skills such as those used in playing tennis, and remembers (or fails to remember) episodic and semantic information''.}

   
  Also, given that context plays such a major role in all language processing (cf. e.g. \citealt{Baker2006}, \citealt[34, 45]{Halverson2013}), it is not inconceivable that a language user should occasionally fail to register something in the message's textual, cultural or other type of environment, which is occasionally of a very complex nature, and instead concentrate on the contents of the message itself, which may then result in pragmatic infelicities. For example, understanding a question such as \textit{Could you open the window (please)}? as a request rather than as a question requires that a person decoding this piece of language, as used in context, pay sufficient attention to the non-linguistic aspects of the utterance as well.


\section{Decision-making: Selecting between the alternatives available}\label{hietaranta:sec:2}

Assuming that language is crucially dynamic in character in the sense of the preceding section, it follows that language users continuously need to decide which particular interpretation they should attach to a language or discourse event that comes their way. Therefore, and also because decision making is an essentially cognitive process, it makes sense to examine this process in some detail since incorrect inferences leading to unintended and thus misconstrued interpretations are also products of bad calls or poor or erroneous decision making.

  The \isi{metaphor} of the human brain as a digital computer is seriously challenged by \citet[616]{Lupyan2013}, who argues -- in consonance with Rosch's \citeyearpar{Rosch1973} prototype view on categories -- that ``the reason people err in classifying items into categories with clear boundaries and known membership criteria is that human categorization algorithms are inherently sensitive to the particulars of the input'', and that ``even in a context that calls for categorical responses, typicality continues to play a role'' \citep[631]{Lupyan2013}. Given this, we must assume that classification is no simple, invariably clear-cut matter which never leaves any residue but that it is rather a process where there are cut-off points but where these points are not fully predetermined -- whence it follows that making sense of the world by means of the process of categorization is a procedure which is not implemented in the same manner by all individuals or even by any one particular individual on all occasions.

  There are two particular issues which I would like to offer as major causes of erroneous decisions and which bear on the consequence of such decisions known as mistranslation. One of them has to do with the possibility of using either a highly conscious, analytical mode of processing information or else an intuitive mechanism crucially based on past experience as suggested by \citet{Berger2007}. The other is the observation that ``in choices between uncertain options, information search can increase the chances of distinguishing good from bad options\ldots ,'' and the associated further observation that ``competition drastically reduces information search prior to choice'' \citep[104]{Phillips2014}. Let us first consider each of the two phenomena in some detail and then see how they help explain the emergence of misunderstandings and possibly also some other poor choices in translation.\footnote{Cf. also Weber \& Johnson's (\citeyear{Weber2009}: 60--62) discussion of the relevance of external search to the task of making choices.}

  \citet[215]{Berger2007} argues that the two modes of rational and experiential information processing ``operate in parallel and synchronically but sometimes one may dominate the other,'' and that ``when information about threatening phenomena [is] presented in statistical or graphical form and require cognitive judgments, the rational system exerts primary influence in determining the nature of the judgment''. Thus, if a person feels that there is a threat which needs to be dealt with, the typical approach is the rational one: we try to analyse the situation rationally and in that way find a way of either removing the threat completely or at least diminishing its effects to what we consider to be a satisfying extent.\footnote{Berger's \citeyearpar{Berger2007} distinction between the rational and experiential modes of thinking is one which is in a number of respects quite similar to the well-known distinction Daniel Kahneman makes between his System 1 and System 2 modes of thinking (for a recent exposition, see \citealt{Kahneman2011}).}

  In \citet{Hietaranta2014}, I suggest that translation tasks can also be viewed as a type of threat in that such tasks are obligations which the translators have to do something about in certain partially predetermined ways to ensure customer satisfaction. Translators need to provide adequate translations, and are thus facing something that will have unfavourable consequences for them if they do not react to the translation situations appropriately. In this sense, then, translators are dealing with a specific type of threat.

  \largerpage[-1]
  What is of the greatest significance in the present context is the observation by \citet[228]{Berger2007} that ``those who are skilled experientially [\ldots]\xspace may [\ldots]\xspace  be able to compensate for the potential shortcomings of the rational system when the task at hand is not amenable to rational analysis [\ldots]''. Given this, one can argue that at least on some occasions it is possible and perhaps even likely that translators may resort, in an unwarranted degree or too hastily, to the experiential mode of processing the information afforded by the source text and that they may therefore fail to detect in the source text something that a more detailed rational analysis would have revealed, whence it in turn may follow that an incorrect choice is made regarding e.g. word choice so that a misunderstanding ultimately occurs.\footnote{I am not suggesting here that the rational way of processing information is in any way necessarily superior to the experiential mode or that the rational mode is always more likely to yield more reliable or more warranted results than those obtained by the experiential mode; see e.g. \citet{Zey1992} for a selection of approaches to cases where well-justified alternatives to rational choices are discussed in detail in a number of different contexts.}

  \largerpage[-1]
  I did not, however, previously consider the possible reason(s) for or cause(s) of such translator behaviour. Here, I would like to suggest the following line of reasoning, based on the discussion of the role of competition in information search by \citet{Phillips2014}, as an explanation for such behaviour.

In essence, \citet[104]{Phillips2014} argue that people tend to apply minimal search strategies as the basis of their choices in cases where it is obvious or even likely that there are competitors around who ``may seize the better option while one is still engaged in search'', and that people may not behave in this manner when there is lack of competition. Thus, ``on a slow shopping day, the leisurely shopper ...  can take his time deciding whether or not to buy the television ...''  while ``on a frantic shopping day like Black Friday, the same shopper is likely to behave very differently''. I agree with this analysis, and would in fact like to propose that it is possible to extend the analysis also to other types of situations where the notion of competition figures just as prominently even if it manifests itself in a somewhat different way.

Specifically, it seems that competition is a special case of the more general notion of psychological pressure (cf. \cite{kilduff2010}), and that translation situations are characteristically cases where the translator has to work under some pressure at least in the sense of translators having to compete against time to meet the deadlines set for the translation commissions and also in the sense of each individual translator having to compete against herself or himself to meet or even surpass the customer's demands and expectations.

As an example, consider the case of a translator having to meet the deadline set for a translation project. If the translator realises that the deadline is approaching and that there is still some checking or research to be done before a finished product can be supplied to the customer, it is understandable that shortcuts may be taken wherever possible, otherwise the translator's attempt to guard her or his reputation as a punctual service provider will suffer. It is here, then, that the possibility of deciding to resort to the intuition-based experiential mode of information processing comes in handy: if you sincerely believe that you have a reasonable amount of experience regarding the type of project you are currently engaged in, your brain may revert, this decision never being subjected to careful, rational analysis, to the experiential mode, and start acting the way it did previously in what you believe was a comparable situation. Specifically, you will then start making decisions of the kind you made previously also, and decide e.g. to use, as a translator, the equivalent you used on the earlier occasion(s). What your brain may not be alerted to, however, is the possibility that the current situation is not fully similar in all essential respects to the earlier situation(s) after all, and that problems may therefore crop up.

Further, if we agree with \citet[59]{Weber2009} when they argue that ``emotions experienced by the decision maker, in addition to the many cognitive factors mentioned\ldots , focus attention on features of the environment that matter for emotion-appropriate action tendencies,'' it makes sense to assume that a translator willing to retain a reputation as a high-quality professional capable of meeting set deadlines may be tempted to resort to shortcuts aiding the relevant cognitive processes even when such shortcuts might be deemed somewhat risky.\footnote{What I argue here seems to be fully in line with and in fact supported by Kahneman's (\citeyear{Kahneman2011}) remarks at the very beginning of the chapter entitled \textit{Cognitive Ease}: ``Whenever you are conscious, and perhaps even when you are not, multiple computations are going on in your brain, which maintain and update current answers to some key questions: Is anything new going on? Is there a threat? Are things going well? Should my attention be redirected? Is more effort needed for this task?''}

To substantiate the above argumentation, let us examine two genuine cases of incorrect choice in the realm of translation and see how they might be explained by reference to what has been said above.

\section{Mr Murphy at work: Selecting the wrong reading}\label{hietaranta:sec:3}

The first case is a mistranslation which I discuss briefly in Hietaranta, where I suggest what I now believe is only a partial explanation. In short, the case concerns a mistake which was made in 2010 in the Finnish translation of a cookbook and which was of such a serious nature that the Finnish Safety and Chemicals Agency \textit{Tukes} (\url{http://www.tukes.fi/en/}) decided to issue a warning on its web site to inform the general public about this potentially hazardous translation mistake. Here are the essential details of the case. The book contained a recipe where a reference was made to a mushroom whose English name is \textit{morel} but which was incorrectly translated into Finnish by means of the Finnish word \textit{korvasieni}. This expression, however, is not the Finnish word for \textit{morel} but rather for another English expression which also refers to a mushroom but to a mushroom which is extremely poisonous unless it is carefully prepared according to specific instructions, viz. \textit{false morel}. The interesting question here is then why the translator made the wrong choice in the first place.

  In Hietaranta (forthcoming) I argue that the translator behaved as follows. As soon as the translator recognises the English word \textit{morel} in the context in question, it becomes necessary to find a Finnish equivalent to the word. Because of the type of the context (a mushroom in a recipe), a number of potential equivalents -- names of edible mushrooms -- are activated in the translator's mind. Next, the number of these candidates needs to be cut down so that a suitable equivalent can be found and used in the translation. At this stage, a mistake is made: the translator realises that the item \textit{morel} has to be linked with a Finnish item which equally refers to an edible mushroom, but what the translator does not realise is that the typicality effect and \isi{cognitive economy} (cf. \citealt[169, 260]{Evans2007}), without any authorisation from the rational mode, insidiously take over and more or less automatically activate in the somewhat unwary translator's mind an equivalent which is deceptively similar to the source text item and which the translator therefore accepts. That something like this may happen is due first and foremost to the fact that the English \textit{morel} is known to be an expression which is reasonably common in mushroom terminology and which further refers to an edible mushroom. On the Finnish side, it is the similarity to these qualities which makes the translator accept the equivalent that first comes to mind: a relatively common item, refers to an edible mushroom, and is connected to the word \textit{morel}. What the translator does not realise is that the Finnish item \textit{korvasieni} is not linked with the English \textit{morel} but rather with the spuriously similar \textit{false morel}.

  
  I now believe that while the argumentation above is valid in all essential respects, it fails to capture the true nature of certain aspects of the decision making involved and thus does not provide an entirely accurate picture of the case. Specifically, my previous argumentation does not seem to capture very well the fact, referred to, in a totally different context, by \citet[18]{Sjoberg2003}, that ``it has been found that extensive training for expertise in a substantial area gives rise to a semi-automatic mode of functioning, which could be equated with intuition\ldots ''. Thus it is quite likely that, in the case of translators as well, routines established via considerable amounts of previous experience tend to conspire to the effect that tried solutions are every now and then resorted to without a sufficient amount of explicit analysis. And once this happens, it may turn out that the new situation or problem requiring a solution is not after all quite so similar to the ones previously considered and solved, whence it follows that translators occasionally make mistakes as a result of applying inadequate amounts of conscious analysis to the case at hand.

  This also seems to accord quite well with Kiel's (\citeyear{Kiel2003}: 671) observation to the effect that ``overconfidence is also found in studies of text comprehension, in which people often do not detect their own failures to understand a passage of text\ldots''.

   Furthermore, if we agree with \citet[359]{Baars2010} when they argue that the human brain typically operates not on discrete picture level representations of the world around us but rather on ``visual images that are prototypical reminders of categories in the world'', it can be argued that our brain occasionally simply fails to single out the correct representative of a class, be it a mushroom or something else, and instead picks out something which is referentially and semantically close enough even if the entity picked out is ultimately not quite what we want. 

  In sum, while \isi{cognitive economy} is a principle which enables us to ensure to some extent that there is in all likelihood always a certain amount of processing capacity available to handle emergencies, it is also the case that economising, especially in stressful situations, on that capacity by resorting to intuition to an unwarranted extent in what must be regarded in hindsight as the wrong place may lead to problems.

  The second case that I wish to consider here briefly is one that I have also considered before in another connection but which I now believe did not receive a fully adequate analysis in my previous treatment of the issue (see \citealt{Hietaranta2014}). In essence, the case is as follows. In December 2011, the news site of the Finnish Broadcasting Company YLE published a news item which informed the readers that according to ``Interior Minister Päivi Räsänen, the cargo manifest of the M/S Thor Liberty on which 69 Patriot ground-to-air missiles were found listed the weapons as `fireworks'. On Friday, officials announced the missiles were listed on the manifest as `rockets', which Kotka port officials had misinterpreted as meaning `fireworks' '' (\url{http://yle.fi/uutiset/missiles\_listed\_as\_fireworks\_on\_ship\_manifest/5472045}). Again, the question is why the incorrect choice was made: out of two equally sensible readings, why was the wrong one chosen?

  My earlier explanation was that ``it is possible and probably even likely that someone whose mother tongue is \ili{Swedish} read the list and by mistake connected \textit{rockets} with the rather similar \ili{Swedish} expression \textit{raket}, which has the two meanings of `missile' and `firework', and that the person simply made the wrong choice''. My tentative explanation was ``that our brain -- despite our best efforts -- occasionally interprets the circumstances we are working under as threatening to the extent that the brain, without asking for a permission from our conscious self, initiates a procedure which will not be halted until a solution has been found. This in turn is so energy-consuming that that the brain will not be able to sustain such a state very long, and so a solution is as it were is forced upon us\ldots'' (ibid.).

  I am no longer sure that this is all that there is to the issue; rather, it seems that the brain is tempted to prefer certain types of readings over other readings in certain types of contexts. Specifically, assuming that contexts work on a principle of exclusion (readings of textual items which are not compatible with those of other items and the text in its entirety are discarded so that the intended reading ultimately surfaces), it may be that the brain for some reason fails to exclude a reading because it does appear to be compatible with the rest of the text even though it is not the only reading with that property. That is, if there are several seemingly appropriate readings available, our processing machinery accepts the reading that first suggests itself even if the reading is not the one intended by the sender of the message. This in turn may have quite a lot to do with (what I believe is) the fact that the brain tends to process contextual clues in specific ways and that the brain sometimes misjudges the relative significance of a particular clue and draws a hasty conclusion, either overestimating or underestimating some aspect of the context (cf. e.g. \citealt[169--193]{Dascal2003}). In translation, some of the consequences of such unwarranted conclusions may then manifest themselves as translation errors of one kind or another.

  As regards the above case of \textit{fireworks} vs. \textit{missiles}, the incorrect initial interpretation probably surfaced because the fireworks reading was so much more likely than the missile reading that the person reading the text just could not stretch his or her imagination to extend the processing cycle to cover missiles too. This in turn was most likely so because in most non-military circumstances missiles seem to be encountered far less frequently than fireworks.

\section{Explaining translation mistakes further: Judgment and decision making in a wider perspective}\label{hietaranta:sec:4}
\largerpage
It thus seems that incorrect choices or bad calls are occasionally made by the human brain because some aspect(s) of a text's cultural, textual or cognitive environment are overlooked so that the reading the brain settles on is not fully backed up by what is available, on closer scrutiny, in the situation. Why such oversights occur is a question which is most likely to have several answers of different orientations. However, it seems that quite a few of these answers or explanations, which are mutually distinguishable from each other, also share some common features.

  One unifying property here seems to be cognition. Since all information entering the brain needs to be processed if the information plays a part in a person's conscious decision making and the ensuing conclusions, it follows that any piece of information which, for whatever reason(s), does not receive the attention it should, may prompt inconsequential inferences, which in turn may instigate associated actions we, on a more detailed analysis, would not be willing to subscribe to.

  Against this background, consider following three points about decision making made by \citet{Weber2009} and their relevance to translation mistakes.

  First, the above hypothesis about the tendency to use the experiential mode of information processing as a shortcut to reach a state where the brain is again reasonably relaxed and in particular has enough reserve processing capacity in case an emergency of some kind should appear seems to receive some support from \citet[66]{Weber2009}, who refer to the good mood maintenance hypothesis, which “assumes that people in a good mood would like to maintain this pleasant state and thus try to avoid hard, analytic work and use cognitive shortcuts instead” (for an earlier formulation of the idea, see \cite[1122, 1128-1130]{Isen1987}). What is most remarkable about this hypothesis and about the evidence accumulated in support of it is the fact that in the experiments conducted it was established that ``good mood resulted in inferior performance and overconfidence'' while ``bad mood resulted in more accurate decisions\ldots'' (ibid.). It is therefore quite likely that a translator who is, perhaps even unconsciously, tempted to use previous experience as recalled by memory as a shortcut solution to the problem of finding a translation equivalent may indeed end up making a translation mistake since the potential translation equivalents may not be analysed and evaluated in sufficient detail.

  This is also in consonance with Weber and Johnson's (\citeyear{Weber2009}: 67) observation that ``both cognitive ...  and affective processes ...  have been shown to influence people's evaluative judgments''. It seems fair to assume that a translator working for commercial goals is constantly under more or less pressure and that the translator's decision making is therefore influenced by this affect.

  Secondly, let us consider the translational relevance of Weber and Johnson's (\citeyear{Weber2009}: 70) argument to the effect that ``social norms dictate the use of different decision principles in different domains (e.g., moral vs. business decisions\ldots)''. Assuming that this is so in general, it seems that we are provided with yet another explanation for the emergence of mistakes in translations for the following reason. If a translator is seeking to comply with the moral code or the principles of the work ethics of the profession to secure the quality of the finished product, he or she will need to expend some effort to make sure that there will be no mistakes in the translation, which obviously requires that a certain amount of time will have to be spent on checking the quality of the final product.

  However, on the other hand it is equally clear that business will not flourish if too much time is spent on any individual stage of a translation project; spending too much time on a project means that the project will not be an economically profitable one, which is of course unacceptable as long as the translator is translating for a living.

  Thus, it is vital that the translator strike a balance between the need to use enough time on any individual project, on the one hand, and the need to economise judiciously on time, on the other. This, in turn, is tantamount to being in a situation where a decision must be made so that the moral norm of \isi{translation quality} is not violated while the business norm of profitability is also observed to a sufficient extent at the same time. In such a situation the translator is then facing a dilemma whose solution necessarily calls for a reconciliation between potentially conflicting types of norms, which is yet another example of a situation where the notion of (psychological) pressure apparently has a part to play in the sense that if the pressure becomes overwhelming, the translator's brain may again inadvertently resort to a shortcut and simply force a solution on the situation so that the level of uncertainty and indeterminacy can be diminished sufficiently and a minimally acceptable level of processing capacity secured (for some relevant discussion of some of the details pertaining to uncertainty and indeterminacy management, see e.g. \citealt{Gudykunst1988, Angelone2010}). If this happens, it should come as no great surprise that a mistake may be made, one or another aspect of the \isi{translation process} having been subjected to insufficient analysis and some subsequent ill-founded decision making.

  Thirdly, the type of analysis advocated in the present paper also seems to receive some support from the fact that \citet[72]{Weber2009}, too, independently argue for a view on decision making which, on nonlinguistic grounds, appears to be fully compatible with what has been argued above. Specifically, Weber and Johnson note that ``individual and cultural differences in decision making seem to be mediated by two classes of variables: (a) chronic differences in values and goals, ...  and (b) differences in reliance on different automatic versus controlled processes, related to cognitive capacity, education, or experience''. Thus, in a vein similar to \citet{Berger2007}, Weber and Johnson here invoke the notion of automatic or less analytic behaviour as an explanatory factor in their account of differences in decision making between individuals and cultures, and on this basis also one can then argue that translation mistakes are sometimes committed because not all translators are always capable of retaining an analytic approach to their tasks even if such a method is required; rather, it seems that some translators may at least temporarily discard the analytic mode of information processing and instead attempt to construct at least part of the translation by means of techniques which are essentially experiential in nature.

  That this is a plausible explanation is a claim which is supported by Weber and Johnson's (\citeyear{Weber2009}: 73) observation, based on the cognitive reflection test, that ``normative choice models may turn out to be descriptive for at least a subset of the general population, those who have a greater ability or inclination to use rational/analytic processing in their decisions''. That is, among translators too there are people who are more prone to analytical thinking and information processing than others, which is one reason why not all translators make the same (kinds of) mistakes when translating the same texts. Those with less patience and more willingness to take risks may well make mistakes which the more analytically minded and cautious colleague may be able to avoid.

\section{A methodological note: Emotions and cognition}\label{hietaranta:sec:5}

The explanations propounded for the two cases of misunderstanding discussed above in \sectref{hietaranta:sec:3} undeniably presuppose that cognition is not an independent human faculty but rather that there are inherent connections between emotions and cognition, in particular. Otherwise it will not be possible to argue that the brain may be likely to prefer certain types of readings or interpretations in certain types of contexts. This is so because in many cases the type of context is determined to a crucial extent by factors connected with emotions; for instance, arguing that the desire to save face for professional reasons is a valid explanatory factor when explaining misunderstandings in translation only makes sense if we assume that emotions affect cognition, the desire to save face being by definition an emotional concept. Thus emotional concepts frequently though not invariably constitute a necessary part of the explanatory apparatus we need to invoke when we try to make sense of what happens in cases of misunderstanding such as the ones discussed above.

While there are a variety of accounts of emotions as regards the ways they affect language use and cognition (cf. e.g. \citealt{Power2006}), there are a number of empirical findings which suggest that it is the complex interplay of different emotions which accounts for much of how our cognition relates to and deals with emotions when we use language. Further, and most importantly for our present concerns, it appears that it is group-level phenomena which particularly shape individuals', including translators', ways of dealing with emotions as they connect with human cognition (cf. \citealt{Kleef2016}).

Thus, given that a translator working for a commercial employer is frequently working under at least somewhat demanding conditions e.g. when the deadline is approaching, it is more than likely that some of the decisions the translator makes are also affected by factors which are inherently coupled with group-level phenomena. This is so because translators are nowadays typically connected with other translators and experts via different types of networks connecting groups of people (cf. e.g. \citealt{Tyulenev2014}), and are therefore also affected in their decision making by other translators and experts. Among these group-level phenomena influencing translators' decision making procedures and the ensuing (mis)interpretations of the source texts they are working on, at least two specific types are of methodological significance.

First, as argued by \citet[710]{Power2006}, ``basic emotions can become `coupled' with each other'' so that emotions may and occasionally in fact do influence each other, which makes it very difficult to analyse emotions in completely unequivocal terms. The specific case of the two emotions of anger and disgust examined by Power is a relevant example in the present context also in that in Power's study the emotion term \textit{disgust} was found to correlate ``more highly with the `Anger' basic emotion scale than it did with its own predicted `Disgust' scale''.

Therefore, if a translator becomes disgusted even by some minor problem(s) with a translation project, it may happen that the translator is also angered through frustration, for instance when an adequate translation equivalent does not present itself to the brain quickly enough. In such circumstances, the translator may then very well accept the first sensible alternative which becomes available after some deliberation. But as we have already seen, there may be several sensible alternatives available, and on second thought it may turn out that it is not the first candidate that the translator should choose. The problem is that the second thought may never materialise itself.

Secondly, in the case of our latter example of the English rocket, group-level phenomena may equally have come into play. Assuming that the brain utilises contextual information in language processing to exclude readings of textual fragments which are incompatible with the remainder and the whole of the text, it may still happen that the brain fails to exclude a sensible reading of part of the text because the reading in question is preferable for emotional reasons with regard to \isi{cognitive economy}. Note the time of the misunderstanding in the fireworks/missile case: according to the news item, the misunderstanding took place on 21 December 2011, that is, just a few days before Christmas and just ten days before the new year. In such emotionally loaded circumstances, it is quite likely that a person who is not a \isi{professional translator} but who is still in need of a domestic equivalent to a foreign expression will readily settle for the first sensible option that comes his or her way. Now, what would be the most likely candidate for such a first sensible equivalent? Since both Christmas and New Year's Eve are times when fireworks are regularly used and seen virtually everywhere in Finland as well as in many other countries, it is quite understandable that it should be the firework interpretation which first came to the mind of the unfortunate layperson translator.

Methodologically, it is of special significance in the present context that the accounts provided above clearly rely on the assumption that the human cognitive systems -- particularly as they are related to decision making -- are not independent of our emotional states but rather both affect them and are affected by them. For this assumption, there seems to exist a body of data which is reasonably convincing both in terms of its quantity and its breadth. Let us consider here just one piece of evidence which relates directly to single word processing since the examples considered above involve single lexical items.

If \citet{Vinson2014} are correct to argue that ``even single words in isolation can evoke strong emotional reactions'' (737) and that ``both negatively and positively valenced items are relevant to survival and well-being'' (744), it is more than likely that lexical items such as \textit{morel} and \textit{fireworks} were both submitted in their respective contexts to processing cycles where emotions were crucially involved for reasons related to existing time limitations and the eventually ensuing risk of face loss if not for anything else.

In sum, then, it seems fair to say that human cognition is affected by emotions in a variety of ways, and that the argumentation of the present paper is thus based on solid grounds as regards the use of emotions as an explanatory factor in the analysis of the cases discussed above.

\section{A final note: Cognition in communication}\label{hietaranta:sec:6}

That translation mistakes are committed in the first place is an observation which is worth subjecting to closer scrutiny because communication in general is relatively smooth and unproblematic even if it does break down occasionally in one way or another (cf. e.g. \citealt{Bosco2006}). Because of the additional linguistic load on communication through translation one might hypothesise that there is something additional about the combination of languages involved in translation which makes it exceptionally hard for human cognition to handle and that the process of communication therefore sometimes breaks down in the form of translation mistakes, in particular (cf. e.g. \citealt[17]{Angelone2010}).

  Above, I have offered, with special reference to the process of decision making, what I believe are some cognitively motivated explanations for such mistakes. Now, to conclude the discussion, I propose to tie up the analysis of translation mistakes in the decision making \isi{frame} with the larger picture of communication in general.

  
  Assuming that \citet{Piller2010}, among others, is correct to argue that cultural factors have a notable effect especially on the pragmatics of human interaction, it makes sense to assume that at least some of the difficulties and problems surrounding translation are dependent on if not caused by cultural differences. Yet, there are clearly other types of problems involved too, and some of these non-cultural difficulties are arguably related to communication in general even if the specific framework assumed is that of translation. Thus \citet{Angelone2010}, who specifically discusses the effects of uncertainty and \isi{uncertainty management} on translation-related problem solving tasks, argues that ``the translation task is essentially a chain of decision-making activities relying on multiple, interconnected sequences of problem solving behavior for successful task completion'' (17), and that difficulties in translation typically lead to uncertainty: ``Should the translator's declarative or procedural knowledge begin to falter at the point of difficulty, \textit{uncertainty} will likely emerge shortly thereafter'', uncertainty being defined as ``a cognitive state of indecision that may be marked by a distinct class of behaviors occurring during the \isi{translation process}'' (18) (italics in the original -- PH).

  
  What is of special significance for our present concerns is Angelone's observation that, for any successful completion of a translation task involving problems, ``in addition to solution evaluation, two other fundamental \isi{uncertainty management} problem solving strategies must be considered, \textit{problem recognition} and \textit{solution proposal}'' (20, italics in the original -- PH). For it is clear that a problem can be either solved or even avoided completely only if it is recognised in the first place; a problem cannot be solved if it is not known to exist.\footnote{The problem may still exist even if it is not known to exist, only it may not surface until it starts having visible consequences as in the case of mistakes in translation causing undesirable recipient behaviour, e.g. people cooking meals which cause food poisoning because of a mistake in the translation of a recipe; for an example in a Finnish context see \url{http://www.tukes.fi/fi/Ajankohtaista/Tiedotteet/Kuluttajaturvallisuus/Ruokaohjeiden-kaannosvirheet-voivat-aiheuttaa-myrkytysvaaran/}.}

  
Thus a translation mistake may be avoided if the translator is aware that a particular type of text in a given type of context is only seemingly easy to translate and is in fact deceptively transparent as far as its translation is concerned. That is, an existing problem can be solved as soon as it is recognised as one. In this sense, translation problems are similar to a number of other types of communication difficulties: a remedy can be administered only if the disease is first diagnosed properly.


  As regards people's success in coping with \isi{uncertainty management} related to translation tasks, it is quite revealing to note Angelone's findings about what really counted in the experiments he conducted on \isi{uncertainty management} involving translation: ``When this study was first conceptualized, assumptions were made that expertise in \isi{uncertainty management} would be revealed by more frequent expression of metacognitive indicators (such as direct articulation) on the part of the professional. While that assumption, as far as it went, was supported, the fact is the mere quantity of metacognition or articulation was not the indicator of expertise and improved performance. Clearly, the manner in which \isi{uncertainty management} unfolded was a much more reliable indicator of potential success in translation than the simple frequency of \isi{uncertainty management} behaviors'' (37). Assuming that this is a valid and reliable result, we can then argue that some mistakes in translation are explainable at least to some extent by reference to the different techniques different translators employ while practicing their profession: those that are capable of solving decision making problems in ways which include adequate amounts and appropriate types of problem articulation (the problem is formulated in sufficiently explicit terms) and solution evaluation (the solution entertained is checked by reliable and appropriate means) make fewer mistakes than those whose metacognitive competence is of the kind which is less suitable for translation work.

  
  This conclusion, it seems, is fully in line with the conception of translations as attempts to solve problems: a source text (or to use \citet[37]{Halverson2013} more chronologically oriented term \textit{anterior text}) which needs to be translated can be construed as a kind of a problem, and just as (especially larger) problems are typically solved step by step, a translation task is also accomplished in stages. In this view, translation is not unlike monolingual communication, where the goal of getting oneself understood by means of a linguistic message or that of understanding someone else can also be viewed as problems awaiting solutions. And just as the implementation of any plan drawn for the purpose of reaching a specific (type of) goal needs to be executed in steps, so is a translation task essentially, because of the decision making operations required, a problem solving task which enables the translator to reach the desired goal at an acceptable cost only if the sum total of the efforts expended on the project includes some minimal number of relevant cognitive operations, which is not always clearly specified or perhaps even specifiable.

  
  Yet, as noted by \citet[39]{Halverson2013}, it is clear that in intercultural communication in general, whether the communication is monolingual or multilingual, there is less common ground than in monocultural communication: ``It is obvious ... that in intercultural communication, whether monolingual, bilingual, or multilingual, the assumption of shared lived environments does not hold: the knowledge bases [of the participants] that may be activated and evoked, as well as the conventional paths of inference, may differ widely''. Since translation is inherently an intercultural operation, it is thus only to be expected that there will be problems in translation, the translator always being natively familiar, through his or her mother tongue competence and its associated culture, with the inner workings of one culture only, whence it follows, in particular, that it is the ``conventional paths of inference'' of that culture only which are readily available to the translator for decision making purposes and other high-level cognitive tasks. In this light, a translation mistake is therefore sometimes caused by the fact that the translator is unable to come up with a translation which would enable the recipients to process the translation by using their own domestic inference machine because of the way the translation is constructed.

  
  This obviously accounts for the fact that recipients sometimes find parts of translation difficult to understand but it also accounts for the possibility of translation mistakes in the following sense. If the translation is constructed in a manner which reflects the translator's inadvertent or unconscious decision to reflect the inference types available in his or her native language and source texts e.g. by means of lexicalisation where no corresponding technique is available in the \isi{target language} text, the end result in the form of the finished translation may ultimately contain forms and expressions which are not entirely natural in or native to the \isi{target language} communication but rather betray their foreign origin via the inference mechanisms required for their successful processing (cf. \citealt[773]{Weigand1999}). The translation sounds ``odd''.

 \largerpage
  That this unnaturalness is a general phenomenon detectable in communication in general and not limited to less felicitously translated texts (translationese) is suggested in a number of studies. Thus it is argued e.g. by \citet[223]{Mustajoki2012} that ``communicative ability includes not only a repertoire of words and structures of the language concerned, but also the skills that are needed to use them in various contexts\ldots''. Therefore, it does not really matter as far as success or failure in communication is concerned whether a person is lacking in communicative competence in a monolingual or in a multilingual context: in either type of context, such a lack is liable to create problems.

  
  But it is not only a lack of competence which may lead to difficulties in communication. What is most important in the present connection is Mustajoki's further argumentation (229), on the basis of the experiments conducted by \citet{Keyzar2002}, to the effect that ``people are systematically biased to think that they are understood when they are not\ldots''. That is, when we perceive no distinct differences between ourselves and others, we tend to use ourselves as the relevant yardstick and reference point. Consider what this means as regards translation. If a translator is gauging the cultural and/or linguistic distance between his or her own culture (language) and that of the future readership and cannot detect, on a particular dimension or even more generally, any clear difference between the source text environment and the target text environment, he or she is likely to rely, even unconsciously, on the cultural and linguistic practices of his or her own culture, and may thereby force some of the associated inference mechanisms on the readers, thereby either making it difficult for the readers to decipher the translation in its entirety or producing a translation which is not adequate or which may even contain one or more translation mistakes.

  
  Also, as regards translators themselves as human agents, it is likely that they may genuinely believe that they have understood the text they are translating when in fact they have not. This is so because ``one basic desire of people is the wish to be regarded as smart and intelligent'' \citep[230]{Mustajoki2012}. If this is so, it is more than likely that there are translators out there who sometimes do not dare to admit to themselves that they have not understood a text they are supposed to translate; if they did, they might lose their face in front of their customer(s). Such intellectual dishonesty towards oneself is of course virtually bound to lead to translation mistakes sooner or later -- most likely, sooner.

  
  Given, then, that translation mistakes are a special case of the more general phenomenon of communication failure, we can say that there are both personal factors and social conditions which contribute to the occurrence of translation mistakes. On the one hand, overconfidence in one's abilities combined with the desire to make a favourable impression on other language users is a likely explanatory factor in some cases of miscommunication involving translation errors of one kind or another. On the other, there are also culturally instilled conditions which are conducive to making translators susceptible to translation mistakes; since familiarity breeds contempt, similarity on one conceptual dimension or other is occasionally taken for sameness or equivalence, and a translation mistake is consequently committed.

  
  In sum, what is argued here seems to agree with most of what is suggested by Mustajoki's \citeyearpar{Mustajoki2012} general account of communication problems, where the notions of \isi{cognitive economy} (the tendency to avoid processing efforts which are not deemed vital) and \isi{mental world} (which includes a person's cultural background, internalized cognitive patterns, emotional states, and various situational factors) play major explanatory roles. With these notions, it appears to be possible to account for a relatively large number of different types of translation mistakes and problems in the manner illustrated above.

\sloppy  
\printbibliography[heading=subbibliography,notkeyword=this]
\end{document}
