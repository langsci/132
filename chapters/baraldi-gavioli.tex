\documentclass[output=paper]{LSP/langsci} 
\ChapterDOI{10.5281/zenodo.1090984}
\title{On the achievement of question-answer sequences in interpreter-mediated interactions in healthcare: Some notes on coordination as mediation}
\author{Claudio Baraldi\lastand Laura Gavioli\affiliation{University of Modena and Reggio Emilia, Italy}}
\captionsetup[figure]{name=Extract} % set back to normal after end of article
\abstract{Following Wadensjö’s well-known concept of coordination (\citeyear[6]{Wadensjo1998}), we draw a distinction between what we have called basic and reflexive coordination. While basic coordination refers to unproblematic rendering of utterances where no communication problem is explicitly addressed, reflexive coordination highlights the process through which actions become relevant that make the participants’ interpretation of what is going on in communication explicit and observable to the other participants. Reflexive coordination is thus the process of communicating about communication. While reflexivity is a characteristic of all types of communication, in interpreter-mediated talk it largely accounts for what is referred to as mediation. 
Here we look at naturally occurring audio-recorded data of doctor-patient interactions where a mediator participates with the function of providing bilingual interpreting service. We focus on sequences which include doctors’ history-taking questions up to the patients’ answers. Our data show that there are three sets of problems mediators need to deal with when translating history-taking questions. First, they need to address not only the content of the doctor’s question, but also the purpose that is projected through that question. Second, they need to re-design the doctor’s question in a way that it is likely to be understood and reacted to appropriately, by the patient. Third, they need to formulate the rendition of the patients’ response for the doctor, in a way as to allow transition to the doctor’s next question or to the conclusion of the history-taking session. Addressing these problems and clarifying the issues related to them, to the participants in the interaction, involves forms of reflexive coordination and mediation. While the analysis of the renditions of history-taking questions does not cover all  forms of reflexive coordination in the data, it is interesting to see how reflexive mechanisms of communication work inside this specific sequences.}
\shorttitlerunninghead{On the achievement of question-answer sequences}
\maketitle
\begin{document}


\section{Introduction}

In the last 20 years, migration fluxes have changed the distribution of the population in Europe, enhancing the \isi{construction} of multilingual societies. One of the consequences of this rapid change is that public institutions, like courts, hospitals and schools, have increasingly served people of varying provenances, speaking different languages. Institutional encounters mediated by bilingual professionals, helping providers and laypeople understand each other, have thus become overextended.  In this context, the importance of understanding the effectiveness of interpreted talk  has attracted the attention of many institutions. For their crucial importance for the population welfare, healthcare services were among the first to  raise interest in language and social research (e.g. \citealt{Angelelli2004Medical, Baraldi2007, Bolden2000, Davidson2000, Hsieh2007, Tebble1999, ValeroGarces2008}).

The ``translators'' involved in healthcare services have different backgrounds and training, ranging from qualified professional interpreters to ``ad-hoc'' family members or hospital staff \citep{Buhrig2004, Meyer2002}, providing language help occasionally. Although the selection of translators depends on a variety of circumstances (including the language spoken by the patients), a rough distinction can be drawn between countries with a long migration tradition, such as the Anglophone and Northern European countries, and those whose immigration experience is recent. While the former have traditionally relied on professional interpreters \citep{Carr1997, Corsellis2008, Hale2007, Roberts2000}, in Belgium, Italy and Spain, to quote just a few, the personnel in charge of \isi{interpreting} are so-called ``intercultural mediators'' \citep{Lizana2012, Merlini2009, Pittarello2009, Verrept2012}. In \ili{Italian} healthcare settings, in particular, intercultural mediators are preferred to interpreters as they are considered more competent in dealing with different cultural perspectives possibly emerging in communication between healthcare providers and migrant patients. Intercultural mediators thus participate in provider-patient interactions with two main explicit institutional requirements: translating talk between providers and patients, and mediating between their potentially different perspectives and views. 

\largerpage 
In this paper, we look at naturally-occurring interactions, collected in \ili{Italian} healthcare settings, involving a healthcare provider, a patient and an intercultural mediator. We look at the work of mediators with the purpose of highlighting some of the practices they use to manage bilingual talk and ``mediate'' it.  Here we discuss the notions of coordination and \isi{mediation}, taken from the literature. In particular, starting from a distinction between basic and \isi{reflexive coordination}, we suggest that forms of \isi{reflexive coordination} can enhance effective forms of \isi{mediation}  in interpreted talk. 

\section{Interpreting as coordination and mediation}
\largerpage
The notion of coordination was introduced by Wadensjö in her seminal work published in 1998. Wadensjö shows that participants' contributions in the interaction cannot be seen as individual contributions, but they ``make sense together''. Participants in the interaction, including the interpreter, display their understanding of what is going on and thus contribute to make sense of it, in relation to each other's contributions and understanding. According to Wadensjö, interpreters coordinate interactions both implicitly and explicitly. In Wadensjö's  terms, implicit coordination is carried out simply by translating, as the choice of language, in bilingual talk, normally selects the speaker of that language. Explicit coordination is instead carried out through actions which focus openly on the organization of talk or on talk dynamics. Contributions addressed to explicit coordination are not necessarily renditions and include requests for clarification, requests for time to translate, comments on translations, requests to observe the turn taking order, and invitations to start or continue talking \citep[108--110]{Wadensjo1998}. Coordination, then, may include actions which do not mirror the textual form of utterances and turns but, instead, work on communication, making sense of utterances and turns in the context of the interaction and for the speakers involved. Coordination is also at the core of the notion that \isi{interpreting} includes \isi{mediation}. The interplay between coordination and \isi{mediation} is not new in \isi{interpreting} studies and research has shown that interpreters' coordination of bilingual talk construes forms of \isi{mediation} (e.g. \citealt{Angelelli2004Medical, Angelelli2012, Penn2012, Pochhacker2008Interpreting, Wadensjo1998}).

Mediation, intended as coordinating parties with different perspectives, is a concept central to, and possibly developed within, the professional context of conflict \isi{mediation}. In conflict studies, \isi{mediation}  introduces a third perspective in the interaction, with the explicit aim of facilitating communication between the conflicting parties and re-contextualise it into a more positive form of relationship. In particular, two theories can help explain the function of \isi{mediation} in conflict management: the theory of \textit{transformative} \isi{mediation} and the theory of \textit{narrative} \isi{mediation}. The theory of \textit{transformative} \isi{mediation} \citep{Bush1994} suggests  that rather than helping conflicting parties solve the practical problems that created their conflict, \isi{mediation} has the function of transforming their relationship. Transformation of relations requires mediator' actions that promote, on the one hand, the parties' empowerment, i.e. their ability to  express themselves and relate with others; on the other hand, the mutual recognition of validity of their different perspectives, although they do not share them. The theory of \textit{narrative} \isi{mediation} \citep{Winslade2008} argues that \isi{mediation} means transforming the adversarial stories narrated by the parties. Narrative \isi{mediation} aims to construe new narratives of both parties' equal rights and responsibilities, which substitute each party's attempt to propose narratives of hegemony, oppression and exclusion of the other party. Combining perspectives derived from these two approaches, conflictive interactions can be handled and transformed either by empowering participation (according to Bush \& Folger), or by promoting new narratives (according to Winslade \& Monk). In order to achieve either mutual empowerment or new narratives, mediators need to coordinate the parties; coordination of talk can thus be oriented to forms of \isi{mediation}. 

Elsewhere \citep{Baraldi2012}, we have reflected on the relationship between coordination and \isi{mediation}, starting from Wadensjö's distinction between implicit and explicit coordination \citeyearpar{Wadensjo1998}. In our contribution, we used a distinction taken from Social Systems Theory \citep{Luhmann1984}, which provided us with a theoretical framework to integrate Wadensjö's implicit/explicit coordination concept into a communication system. Luhmann's distinction is between basic self-reference and reflexivity. Basic self-reference is needed in order to achieve communication \citep[600-601]{Luhmann1984}. It means showing understanding through an utterance, which unproblematically refers to the previous one. A basic self-referential  process of communication is a smooth process in which each utterance refers to the previous one, without analysing or contesting its meaning. Reflexivity is instead involved when participants are engaged in actions that make their interpretation of what is going on in communication explicit and public (\citealt[601]{Luhmann1984}; see also \citealt{Heritage1985, Pearce1980, Weigand2010}), These actions make reference to the participants' perspectives, positions and attitudes and construe their meaning in the interaction. 

On the basis of Luhmann's distinction, coordination of interpreter-mediated interactions can be considered ``basic'' when a turn is dealt with ``smoothly'' and ``unproblematically'' in following talk. This is normally the case when a turn is translated into a next turn. Basic coordination does not depend on the actual complexity of the interpreters' rendition of interlocutors' talk. A rendition can be very complex, but its complexity is ``resolved'' by the \isi{interpreting} professionals in their rendered turns. This means that the rendition is not dealt with in following talk, but it is immediately provided. Reflexive coordination, instead, includes those cases where some \isi{interpreting} problem or issue projected in the turn is addressed and dealt with in following talk. This may be a simple problem, like a problem of hearing (``can you say it again please?''), as mentioned by \citet{Wadensjo1998}, but, more interestingly, it may have to do with the possibility of communication to act reflexively on the communication process in terms of ``what we mean'' and ``what we are doing here'', in this particular interaction or sequence. Reflexive coordination in \isi{interpreting} includes actions like asking for clarification, glossing, commenting and showing understanding actively. 

Discrepancies of understanding or \isi{acceptability} inside mediated communication are thus observable through forms of \isi{reflexive coordination}. These can be forms of talk which get back to the actions that might cause understanding or \isi{acceptability} problems and address them. This talk-reflexive phase can take many forms, but following suggestions from theories of conflict \isi{mediation}, they can be oriented to: (1) displaying an empowering sensitivity for participants' perspectives \citep{Bush1994}, and (2) fostering new narratives that are helpful to promote mutual understanding and acceptance on the part of the participants \citep{Winslade2008}. In this way, \isi{reflexive coordination} constructs forms of \isi{mediation} in the interpreter-mediated interaction. 

In what follows, we shall introduce the empirical part of our research and will look at examples of basic and \isi{reflexive coordination} in \isi{doctor-patient interaction} interpreted by intercultural mediators. We will see that, in the selected data, mediators   display sensitivity for the participants' perspectives and enhance new narratives, as mediators of conflicts are also invited to do. Actual ``conflicts'', however are not visible in the data since participants' (possibly divergent) positions and perspectives are made clear and relevant in the interaction and then treated in reference to the goals of the medical encounter. Mediators here ``interpret'' patients' and doctors' perspectives  in ways that allow participants to understand each other's contributions and react accordingly. In this sense, \isi{mediation} can be considered a particular form of \isi{interpreting} and can be helpful to interpreters and mediators alike, in their work in healthcare.

\section{Data and methods}

The analysis we present in this paper is based on a collection of around 200 consultations involving healthcare providers, migrant patients and intercultural mediators. The data were recorded in maternity/gynaecological settings with \ili{Arabic}-speaking or English-speaking female patients from North and West Afri\-ca. Some data involve male patients and were collected in general practice surgeries. The mediators are all women in their thirties, all with a migration experience. There are three mediators in the English set of data and five in the \ili{Arabic} set. The data were recorded in the course of a long-term research project based on the collaboration between an academic team of researchers and a local healthcare institution, which is one of the most advanced in Italy as to `migrant-friendliness' and services for migrants \citep{Chiarenza2008}. Our data are transcribed following conversation analysis methods \citep{Jefferson1978, Psathas1990}, which provide a graphic representation of some of the most common ``sounds'' of conversation, such as lengthening, ``erms'', ``mhm'' and \isi{pauses}. For \ili{Arabic} data, we used Latin script, not only because it is more easily adaptable to bilingual conversation transcripts (think of e.g. the problem of representing overlap between a left-to-right written \ili{Italian} turn and a right-to-left \ili{Arabic} one), but also because: a. classic \ili{Arabic} is not always appropriate to represent spoken varieties of  {Moroccan} and some  {Moroccan} words do not ``exist'' in classic \ili{Arabic}, b. because the Latin transcript is commonly used by \ili{Arabic} (young) speakers in digital ``pseudo-spoken'' communication, like instant messaging (see e.g. \citealt{Palfreyman2003}). 

The description of the practices discussed here makes reference to conversation analytic studies of doctor-patient interactions, as in e.g. \citet[51--169]{Heritage2010} and \citet{Heritage2006Communication}. In this paper, however, conversation analysis (CA) is combined with other methodological approaches such as those used in conflict \isi{mediation} \citep{Bush1994, Winslade2008} and in social systems theory \citep{Luhmann1984}. According to CA research on \isi{doctor-patient interaction} in monolingual talk, doctors' questions can be split into different types \citep{Heritage2006The} projecting different sets of goals. Two of these types, possibly the most important ones, are ``general inquiry'' questions and ``history-taking'' questions. General inquiry questions ``allow patients to present their concerns in their own terms'' \citep[92]{Heritage2006The} and they are non-focused and open questions \citep{Robinson2001Closing}, like, ``what's your problem?''. History taking questions propose a precise setting of the medical agenda: they constrain patients' responses and are ``closed ended'' \citep[97]{Heritage2006The}, projecting patients' short, e.g. yes/no, answers. Examples of history-taking questions are those about the age or profession of the patients, their life-habits, most significant diseases and the like.

What follows is based on a systematic analysis of those sequences in our corpus which include doctors' history-taking questions up to the patients' answers. Our data show that there are three sets of problems mediators need to deal with when translating history-taking questions. First, they need to address not only the content of the doctor's question, but also the purpose that is projected through that question. Second, they need to re-design the doctor's question in a way that it is likely to be ``taken-up'', i.e. understood and reacted to appropriately, by the patient. Third, they need to formulate the rendition of the patients' response for the doctor, in a way as to allow transition to the doctor's next question or to the conclusion of the history-taking session. Addressing these problems and clarifying the issues related to them to the participants in the interaction, involves forms of \isi{reflexive coordination} and \isi{mediation}. While the analysis of the renditions of history-taking questions does not cover all  forms of \isi{reflexive coordination} in the data, it is interesting to see how reflexive mechanisms of communication work inside this specific sequence. 

\section{Mediators' coordination of history-taking sequences}
\subsection{Basic coordination}

As mentioned above, basic coordination is achieved when the rendition is posed unproblematically, that is, when the mediator translates an utterance or a short series of utterances by posing their contribution as a repetition in the other language of what was said. Let us have a look at two extracts. In \extractref{baraldi-gavioli:extract:1}, both the doctor's question (turn 1) and the patient's reply (turn 3) are rendered immediately in the next turn, with no hesitation. The Doctor's question is slightly summarized and the patient's answer is a bit adjusted showing the mediator's interpretation of a possibly ambiguous patient's turn (\textit{I can't very eat} = ``I can't eat very much'').

\begin{figure}
	\begin{varwidth}{\textwidth}
	\begin{description}[align=left, nosep, style=nextline, leftmargin=3em, format=\normalfont\footnotesize]
	\item [1 \hspace{0.3em} D:] Mangiare, bere, norma- tutto norm[ale? Riesce? \\ \textit{Eating, drinking, norma- all normal? Can he?}
	\item [2 \hspace{0.3em} M:] [Do you: eat (.) normally?	
	\item [3 \hspace{0.3em} P:] Sometimes (I can't very-) (.) eat.
	\item [4 \hspace{0.3em} M:] A volte non ha l'appeti[to. \\ \textit{Sometimes he doesn't have appetite.}
	\item [5 \hspace{0.3em} D:] [Non ha fame. da- sempre da due settimane? \\\textit{He's not feeling hungry. fo- this too for two weeks?}
	\end{description}
    \end{varwidth}
\caption{}
\label{baraldi-gavioli:extract:1}
\end{figure}

\clearpage
Even though the renditions are modified by the mediator in order to make their meaning and function clear, such meaning and function are a matter of the mediator's \isi{interpreting}: they are not posed as a problem, nor do they seem to cause problems in the interaction.

  In \extractref{baraldi-gavioli:extract:2}, we have a similar example. The doctor's question in turn 1 is rendered immediately in the next turn with a repetition of the question and a change of pronoun (from ``she'', used by the doctor, to ``you'' used by the mediator). The patient's reply, repeating the mediators question with a confirming intonation, is summarised in a confirmation answer (``yes'', turn 4), addressing the doctor's question very explicitly.

\begin{figure}
	\begin{varwidth}{\textwidth}
	\begin{description}[align=left, nosep, style=nextline, leftmargin=3em, format=\normalfont\footnotesize]
	\item [1 \hspace{0.3em} D:] E' la prima volta nella sua vita che ha avuto un ritardo? \\ \textit{Is this the first time in her life that her period is late?}
	\item [2 \hspace{0.3em} M:] awwal marra [kai jik had taakhur? \\ \textit{Is this the first time that your period is late?}
	\item [3 \hspace{0.3em} P:]  [awwal marra, dart liya had taakhur. \\ \textit{This is the first time my period is late.}
	\item [4 \hspace{0.3em} M:] Sì. \\ \textit{Yes.}
	\item [5 \hspace{0.3em} D:] Mch. quanti anni ha? \\ \textit{How old is she?}
	\end{description}
    \end{varwidth}
\caption{}
\label{baraldi-gavioli:extract:2}
\end{figure}

These two examples are interesting for a series of reasons, two of which can be mentioned here. First, they show the mediator's understanding of the participants' contributions and may suggest circumstances under which such interpretation can be considered plausible. Second, they show types of renditions (summarised, modified) and may thus lend themselves to reflections about whether these mediators' choices are good or whether they might be improved. So basic coordination is not ``easy'' in mediated talk since rendition choices are inherently ``difficult'' choices. Basic coordination though moves bilingual conversation forward and rendered utterances are unproblematically referred to previous talk, in a process where (possible) communication obstacles are not made observable to the participants in the conversation. 

\subsection{Reflexive coordination} 

Reflexive coordination is involved when actions become relevant that make the participants' interpretation of what is going on in communication explicit and observable to the other participants. In these cases, aspects of communication (problems, goals, perspectives) are raised and dealt with in the interaction. In what follows, we will show three extracts. In all of them, a sometimes apparently very little detail is raised and explored with the patient in a direction that meets the goals projected by the doctor's initial history-taking question. While the ``details'' raised in these interactions may be considered small translating issues, we argue that the ways mediators coordinate talk empowers the patients' and the doctors' perspectives and allows their narratives to be produced in equal ways. This highly sophisticated work is what probably allows the raised ``detail'' to be rapidly treated and solved. 

\begin{figure}
	\begin{varwidth}{\textwidth}
	\begin{description}[align=left, nosep, style=nextline, leftmargin=3em, format=\normalfont\footnotesize]
	\item [53 \hspace{0.3em} D:] Vive qui da sola? \\ \textit{Does she live here alone?}
	\item [54 \hspace{0.3em} M:] Si. \\ \textit{Yes.}
	\item [55 \hspace{0.3em} D:]   Non ha nessuno, [parenti, cugini? \\ \textit{Does she have no one, relatives, cousins?.}
	\item [56 \hspace{0.3em} M:] [You hav- do you live here alone, you don't have brothe::r?
	\item [57 \hspace{0.3em} P:]  °I have a brother.° 
    \item [58 \hspace{0.3em} M:]  (.) Mhm. 
    \item [59 \hspace{0.3em} D:]  [(Non ha -) \\ \textit{She doesn-} 
    \item [60 \hspace{0.3em} M:]  [(Do) you live with your brother?
    \item [61 \hspace{0.3em} P:]  Mh. 
    \item [62 \hspace{0.3em} M:]  Sì vive [col fratello \\ \textit{Yes she lives with her brother}
    \item [63 \hspace{0.3em} D:]  [Vive col fratello, benissimo. \\ \textit{She lives with her brother, very good.}
	\end{description}
    \end{varwidth}
\caption{}
\label{baraldi-gavioli:extract:3}
\end{figure}

In \extractref{baraldi-gavioli:extract:3}, the doctor's history-taking question is posed initially in turn 53 (``does she live here alone?'') and continued in turn 55 (``does she have no one, relatives, cousins?''), treating the mediator's ``yes'' in turn 54 as a continuer. As it is clear from previous turns in the interaction, this question from the doctor is important because the patient has lost a lot of weight and although she does not say it, the doctor believes the cause of this is that she does not eat enough. So he is exploring whether there is someone living with her who may possibly check that she has enough to eat. The mediator renders the doctor's question first in turn 56. In turn 57, the patient answers that she has a brother, but does not clarify whether she lives with him. At this point the mediator suspends her rendition of the patient's turn and provides first a continuer (in turn 58) and then asks a more direct question to the patient who hesitatingly confirms that she lives with her brother. This information is rendered to the doctor, in a ``yes''-format, answering the doctor's initial question. The doctor treats this information as sufficient for the moment (turn 63: ``she lives with her brother, very good'') and plans a complete check-up for the patient (data not shown).  

In the extract, the significance of the doctor's history-taking question is negotiated first, between the mediator and the doctor (turns 53-55) and then rendered to the patient. In her rendition, the mediator addresses the specificity of the doctor's question (``do you \textbf{live} with your brother?'', turn 60), re-designs the doctor's question in order to achieve a patient's appropriate reaction and eventually renders the patient's answer to the doctor in a way as to allow for the doctor's next action. Turn-coordination is thus constructed in a way as to display sensitivity for the participants' perspectives (even if this may produce just a feeble ``mh'' from the patient, as in turn 61) and to foster new narratives that are helpful to produce mutual understanding. By passing to a check-up planning, in fact, not only does the doctor go on with his medical schedule, but he also displays understanding that the patient may be in need of assistance although she hesitates to talk about it (there are in fact no  more inquiries into the patient's life-style and a check-up is planned).

  In \extractref{baraldi-gavioli:extract:4}, we can see doctor's history-taking question in turn 57. In turn 58, the mediator re-designs the question for the patient (``\textbf{have you got} vaginal discharges?''). The patient's reply shows understanding of the word ``discharges'' and the patient describes some discharges she has (``like blood but dark ones'', turn 59). The mediator then explores the patient's answer more in depth as to get a description, from the patient's perspective, that is meaningful in the doctor's perspective and such that it allows the mediator to answer the doctor's question (``brown?'', turn 60). So the colour of the patient's discharges is described in turns 60-62 and rendered to the doctor in turn 63. The Doctor's next question shows understanding of the patient's description and also that the patient may not know how precisely vaginal discharges look like. In her new question in turn 64, then, the doctor clarifies what type of discharges she is asking about, which enables the mediator to explain it better to the patient in turn 65.

\begin{figure} 
	\begin{varwidth}{\textwidth}
	\begin{description}[align=left, nosep, style=nextline, leftmargin=3em, format=\normalfont\footnotesize]
	\item [57 \hspace{0.3em} D:] Pe:rdite:? Vagina:li? \\ \textit{Discharges? Vaginal?}
	\item [58 \hspace{0.3em} M:] Ka duz mnk shi haja? \\ \textit{Have you got vaginal discharges?}
	\item [59 \hspace{0.3em} P:] Kai duz mnni bhal dm wa lakin khl \\ \textit{I have discharges like blood but dark ones}
	\item [60 \hspace{0.3em} M:] =qhwi? \\ \textit{Brown?}
	\item [61 \hspace{0.3em} P:] =ah, [bhal marrone \\ \textit{Yes, they seem brown} 
    \item [62 \hspace{0.3em} M:] [Ah. \\ \textit{Yes.}
    \item [63 \hspace{0.3em} M:] Mhm. Ha delle perdite marroni. \\ \textit{She has brown discharges.}
    \item [64 \hspace{0.3em} D:]  Però perdite tipo bianche de:nse, con prurito o bruciore? \\ \textit{And like white viscouse discharges, itchy or burning?}
    \item [65 \hspace{0.3em} M:]  Ma' byed, qasseh shi shwiya w ka ihrqk shi shwiya? \\ \textit{White discharges, a bit viscouse and that give you some burning feeling?} 
	\end{description}
    \end{varwidth}
\caption{}
\label{baraldi-gavioli:extract:4}
\end{figure}

Here too, we have a rather elaborated coordinating work, where understanding of and perspectives on medical subjects are displayed and made clear by the interlocutors, who are also provided with access to ``the other's'' understanding and perspective. Sensitivity to both the patient's and the doctor's perspectives is shown, which allows for these perspectives to be displayed in talk, and the patient's and the doctor's narratives to be produced. Even though these concern a small detail in the medical encounter, this detail is a very important one for the patient's state of health and needs to be focused on and dealt with adequately in the interaction.

\begin{figure}
	\begin{varwidth}{\textwidth}
	\begin{description}[align=left, nosep, style=nextline, leftmargin=3em, format=\normalfont\footnotesize] 
	\item [1 \hspace{0.3em} D:] Ultima mestruazione quando è stata? \\ \textit{Last menstruation when was it?}
	\item [2 \hspace{0.3em} M:] Akhir marra jatk fiha l `ada shahriya? \\ \textit{Last time you had your period?}
	\item [3 \hspace{0.3em} P:] Rab'awa'ishrin (.) f sh'har juj \\ \textit{Twenty-fourth (.) in the month of February.} (2.0)
	\item [4 \hspace{0.3em} M:] F sh'har juj? \textit{In February?}
	\item [5 \hspace{0.3em} P:] Ah, rab'awa'ishrin (.) f sh'har juj. \\ \textit{Yes, twenty-fourth of February.}
	\item [6 \hspace{0.3em} M:] F sh'har- f had sh'har ma jatksh? \\ \textit{In the month- in this month you didn't have it?}
	\item [7 \hspace{0.3em} P:] Majatnish, yallah jatni, ghlt lik dart liya retard tis' ayyam. \\ \textit{I didn't have, I have just had it, I told you I had a nine-day delay.}
	\item [8 \hspace{0.3em} M:] Yallah jatk? \\ \textit{You've just had it?}
	\item [9 \hspace{0.3em} P:] Ah. \\ \textit{Yes.}
	\item [10 \hspace{0.3em} M:] Imta jatk? \\ \textit{When did you have it?}
	\item [11 \hspace{0.3em} P:] Jatni:: el bareh. \\ \textit{I had it yesterday.}
	\item [12 \hspace{0.3em} M:] Ehm, ya'ni les regles tsamma dyal l bareh mush- \\ \textit{Ehm so yesterday menstruation don't-}
	\item [13 \hspace{0.3em} P:] Ah, ghlt dyal bareh, mashi lli ghlt dak sh'har \\ \textit{Yes I said yesterday, not that from last month.}
	\item [14 \hspace{0.3em} M:] Eh, no, akher marra. ma'natha nti daba haid? \\ \textit{Well no, last time. So you're having your period now?}
	\item [15 \hspace{0.3em} P:] Ah. \\ \textit{Yes}
	\item [16 \hspace{0.3em} M:] Allora, attualmente è mestruata. (.)  Le sono venute ieri. \\ \textit{Well, she's having her period now (.) It came yesterday.}
	\item [17 \hspace{0.3em} D:] Ah! Allora bisogna che torni. \\ \textit{Ah! So she needs to come back.}
	\end{description}
    \end{varwidth}
\caption{}
\label{baraldi-gavioli:extract:5}
\end{figure}

  In \extractref{baraldi-gavioli:extract:5}, we have a slightly different and more problematic case. Here the doctor's question (``Last menstruation when was it?'', turn 1) is rendered immediately in the following turn, by repeating it in \ili{Arabic}. This question apparently poses no problem of understanding on the part of the patient, who answers immediately in turn 3 providing a date (``Twenty-fourth in the month of February''). The date the patient provides, though, refers to a period that is over a month ago. The mediator stops (see pause between turns 3 and 4) and in turn 4 a sequence is opened where understanding of the doctor's question and of the patient's answer are dealt with. It is made clear (turns 4-9) that the patient was referring to the date of the menstruation before the current one, that the patient had her last period on the day before (turns 10-11) and it is also clarified that the doctor's use of the word ``last'' was referred  to the most recent one, \textit{including} the current one (not to the last \textit{before} the current one). This enables the mediator to render a description of the physical conditions of the patient to the doctor, who displays her understanding and her new (consequent) narrative: the patient will have to get back to the surgery after her menstruation, since the pap-test, she was here to take, cannot be taken during patients' menses.

In the above extract, a sensitivity for the participants' perspectives is displayed through the mediator's pause between turns 3 and 4, which signals that there may be something wrong in the current state of understanding in the interaction and the dyadic sequence she engages in with the patient, which explores what is the patient's current understanding of the doctor's question. New narratives are thus fostered, e.g., about the meaning of ``last menstruation'', for the patient and the doctor, and about the ways in which medical procedures are carried out (i.e. a pap-test cannot be taken during menstruation).

\section{Comments and concluding thoughts}

Here, we have discussed a conceptual distinction between what we have called, after \citet{Luhmann1984}, ``basic'' and ``reflexive'' coordination. While basic coordination shows interesting aspects in terms of translation choices, \isi{reflexive coordination} seems to involve forms of \isi{mediation} activity where the mediators need to work on the participants' perspectives and narratives in order to make them relevant and ``treatable'' in the interaction. Let us now conclude with some summarizing points and some final considerations.

  As for the summarizing points, we can probably mention three. First, coordination is a highly contextualized concept, referring both to the general context of the interaction (e.g. the medical context) and the local, sequential one (e.g. history-taking sequences). Here, we have focused on the sequence including history-taking questions up to the rendition of the patient's answer and we have looked at how interlocutors orient in pursuing a relevant patient's answer to a doctor's specific question. We have seen that although these sequences sometimes focus on very small details (who lives with the patient, a precise description of her discharges, the date of her menstruation), getting to the answer may involve a clarification of understanding which includes different perspectives.

\newpage   
Second, clarification of understanding and perspectives can take place in various ways. Here, we have examined some which seem to us effective in that they: a. display the participants' views, b. allow the interlocutors to recognize such views as possibly different from theirs, but as plausible and understandable, and c. react accordingly. Reactions may eventually enhance new narratives. Some that we have seen here have to do with a recognition on the part of the doctor that the patient will not say much more about her problem and it is probably better to examine her physically, with the observation that the patient might not know what ``vaginal discharges'' are like and a more precise question needs to be posed, or with a mutual realization that there may be mismatches about what can be understood with the word ``last'' in a doctor's question about the patient's ``last'' menstruation. 

Third, we have suggested that effectiveness, in the extracts shown, is achieved through a mediator's orientation to: a. display an empowering sensitivity to the participants' perspectives, and b. foster new narratives of the patients' problems. These two orientations are strictly connected. We have seen the ways in which the mediators display sensitivity for the participants' perspectives, capturing the sense and purpose of the doctors' question and re-designing it for the patients, thus helping the patients provide answers in terms that are relevant to those projected by the doctors' questions.

Our final considerations are three. The first one is that while the notions of ``perspective'' and ``narrative'' have often been attributed to cultural specificities or to highly different, possibly deeply separating positions, no such ``differences'' can be observed in the data presented here. The reasons are possibly two. The first has to do with the context we have examined. Doctor-patient interaction is not a highly conflictive setting and doctors and patients normally collaborate in the process of providing/getting care. The second, is in our view more interesting and has to do with the approach to \isi{mediation} that is taken by the bilingual professionals involved in our data. Sensitivity for participants' perspectives in our data is displayed by allowing these perspectives to be expressed and recounted. Interestingly, mediators here do not ``say'' that the patient hesitates, that she does not know what vaginal discharges are, or that she has misunderstood the question about the menstruation date. This emerges in the interaction \textit{from the participants}, who are led and allowed to recognize each other state of understanding and deal with it ``by themselves''. This, we believe, suggests that forms of effective \isi{mediation} have much more to do with the promotion of interlocutors' participation than with the explicit explanation of different positions.

\newpage 
A second final consideration regards basic and \isi{reflexive coordination}. In this paper we have focused on reflexivity and have looked at ways in which forms of \isi{reflexive coordination} can achieve \isi{mediation}. We have also noted that basic coordination has to do with translators' choices that are posed as ``unproblematic'' in the interaction, that is to say, possible problems are solved in translation without further exploration or clarification and without getting back to the participants involved. While the translators' choices are often ``difficult'' and interesting ones and possibly contain forms of \isi{mediation} (what \citealt{Pochhacker2008Interpreting} has called ``linguistic \isi{mediation}''), the dynamics in which participants engage in basic or reflexive forms of coordination is not clear yet and definitely needs further exploration.

The third and last consideration is that \isi{reflexive coordination} can fail in promoting effective communication processes when the  mediators' modified renditions reduce either the doctors' or, most crucially, the patients' opportunities to participate actively in the interaction. Particularly in the case the latter participation is impeded or limited, narrative \isi{construction} fails with consequent disempowerment of the participants in the interaction, especially the patient (see e.g. \citealt{Baraldi2014, Baraldi2008, Bolden2000, Davidson2000}).

In conclusion, the relationship among the notions of \isi{interpreting} and mediating is not an easy-to-establish one and involves active forms of coordination of the interaction. The point is not to define \isi{interpreting} as \isi{mediation} in general terms, but to identify the actual means and actions that characterize \isi{mediation} as a form of \isi{interpreting} activity, displayed in talk coordination. One characteristic of \isi{mediation} we have looked at here is that \isi{mediation} does not necessarily highlight ``intercultural'' forms of talk, even when the participants have different geographical and national origins and even though different perspectives and different states of understanding are issued. Talk-coordinated actions orienting to the promotion of participants' display of understanding and perspectives seem to act quite strongly in the achievement of their recognition by the participants. This allows for new narratives to be produced, which make the participants' perspectives plausible and relevant in the interaction. This, we believe, is the contribution of this paper to the definition of at least one form of \isi{mediation} in interpreter-mediated talk.

\newpage 
\sloppy
\printbibliography[heading=subbibliography,notkeyword=this]

\captionsetup[figure]{name=Figure} %resume figure styling
\end{document}


% @book{Angelelli2004,
% 	address = {Cambridge},
% 	author = {Angelelli, C.},
% 	publisher = {Cambridge University Press},
% 	title = {\textit{Medical Interpreting and Cross-cultural Communication}},
% 	year = {2004}
% }
% 
% @incollection{Angelelli2012,
% 	address = {Amsterdam/Philadelphia},
% 	author = {Angelelli, C},
% 	booktitle = {\textit{Coordinating participation in dialogue interpreting},},
% 	editor = {C. Baraldi and L. Gavioli},
% 	pages = {251-268},
% 	publisher = {John Benjamins},
% 	title = {Challenges in interpreters' coordination of the \isi{construction} of pain},
% 	year = {2012}
% }
% 
% @article{Baraldi2014,
% 	author = {Baraldi, C},
% 	journal = {Languages Culture Mediation},
% 	number = {1-2},
% 	pages = {17-36},
% 	title = {An interactional perspective on \isi{interpreting} as mediation},
% 	volume = {1},
% 	year = {2014}
% }
% 
% @incollection{Baraldi2007,
% 	address = {Amsterdam \& Philadelphia},
% 	author = {Baraldi, C. and Gavioli L},
% 	booktitle = {\textit{Dialogue and Culture},},
% 	editor = {Marion Grein and Edda Weigand},
% 	pages = {155-176},
% 	publisher = {John Benjamins},
% 	title = {Dialogue \isi{interpreting} as intercultural \isi{mediation}: an analysis in healthcare multicultural settings},
% 	year = {2007}
% }
% 
% Baraldi, C. and Gavioli, L. 2008. Cultural presuppositions and re-contextualisation of medical systems in interpreter-mediated interactions. \textit{Curare. Journal of Medical Anthropology} 31(2+3): 193-203.
% 
% @incollection{Baraldi2012,
% 	address = {Amsterdam \& Philadelphia},
% 	author = {Baraldi, C. and Gavioli L},
% 	booktitle = {\textit{Coordinating Participation in Dialogue Interpreting},},
% 	editor = {Claudio Baraldi and Laura Gavioli},
% 	pages = {1-22},
% 	publisher = {John Benjamins},
% 	title = {Introduction: Understanding coordination in interpreter-mediated interaction},
% 	year = {2012}
% }
% 
% 
% @article{Bolden2000,
% 	author = {Bolden, G},
% 	journal = {\textit{Discourse Studies}},
% 	number = {4},
% 	pages = {387-419},
% 	title = {Toward understanding practices of medical \isi{interpreting}: interpreters' involvement in history taking},
% 	volume = {2},
% 	year = {2000}
% }
% 
% \begin{styleList}
% @incollection{Bührig2004,
% 	address = {Amsterdam/Philadelphia},
% 	author = {Bührig, K. and Meyer, B},
% 	booktitle = { \textit{Multilingual Communication (Hamburg Studies on Multilingualism 3},},
% 	editor = {J. House \& J. Rehbein},
% 	pages = {43--62},
% 	publisher = {John Benjamins},
% 	title = {Ad hoc \isi{interpreting} and achievement of communicative purposes in briefings for informed consent},
% 	year = {2004}
% }
% \end{styleList}
% 
% @book{Bush1994,
% 	address = {San Francisco},
% 	author = {Bush B. R. A. and Folger J. P.},
% 	publisher = {Jossey-Bass},
% 	title = {\textit{The Promise of Mediation: Responding to Conflict through Empowerment and Recognition}},
% 	year = {1994}
% }
% 
% @book{Carr1997,
% 	address = {Amsterdam/Philadelphia},
% 	editor = {Carr, S., Roberts, R., Dufour, A., and Steyn D.},
% 	publisher = {John Benjamins},
% 	title = {\textit{The critical link: Interpreters in the community}. \textit{Papers from the 1st International Conference on Interpreting in Legal, Health, and Social Service Settings}},
% 	year = {1997}
% }
% 
% Chiarenza, A. 2008. Towards the development of health systems sensitive to social and cultural diversity. In  \textit{Health and Migration in the EU: conference proceedings}, 167-176. Lisboa: Ministry of Health of the Republic of Portugal.
% 
% @book{Corsellis2008,
% 	address = {Basingstoke},
% 	author = {Corsellis, A.},
% 	publisher = {Palgrave},
% 	title = {\textit{Public service \isi{interpreting}. The first steps}},
% 	year = {2008}
% }
% 
% @article{Davidson2000,
% 	author = {Davidson B},
% 	journal = {`The interpreter as institutional gatekeeper: The social-linguistic role of interpreters in \ili{Spanish}-English medical discourse', \textit{Journal of Sociolinguistics}},
% 	number = {3},
% 	pages = {379-405},
% 	volume = {4},
% 	year = {2000}
% }
% 
% @book{\textstyleMaiuscoletto{\textup{Hale2007,
% 	address = {Basingstoke},
% 	author = {\textstyleMaiuscoletto{\textup{Hale, S. B.},
% 	publisher = {Palgrave}},
% 	title = {} }\textstyleMaiuscoletto{\textit{Community Interpreting}}\textstyleMaiuscoletto{},
% 	year = {2007}
% }
% 
% @incollection{Heritage1985,
% 	address = {London},
% 	author = {Heritage J},
% 	booktitle = {\textit{Handbook of Discourse Analysis, Vol. 3. Discourse and Dialogue},},
% 	editor = {T. Van Dijk},
% 	pages = {95-117},
% 	publisher = {Academic Press},
% 	title = {Analysing News Interviews: Aspects of the Production of Talk for an Overhearing Audience},
% 	year = {1985}
% }
% 
% @incollection{Heritage2009,
% 	address = {New York/London: Routledge.Heritage J., and Maynard D. (eds.) 2006. \textit{Communication in Medical Care: Interactions between Primary Care Physicians and Patients.} Cambridge},
% 	author = {Heritage J},
% 	booktitle = {\textit{Communicating to manage health and illness},},
% 	editor = {D. Brashers and D. Goldsmith},
% 	pages = {147-164},
% 	publisher = {Cambridge University Press},
% 	title = {Negotiating the legitimacy of medical problems: A multi-phase concern for patients and physicians},
% 	year = {2009}
% }
% 
% @article{Heritage2006,
% 	author = {Heritage J. and Robinson J},
% 	journal = {\textit{Health Communication}},
% 	number = {2},
% 	pages = {89-102},
% 	title = {The structure of patients' presenting concerns: physicians' opening questions},
% 	volume = {19},
% 	year = {2006}
% }
% 
% @article{Heritage2010,
% 	author = {Heritage J. and Clayman S},
% 	journal = {\textit{Social Science and Medicine}},
% 	pages = {924-937},
% 	title = {\textit{Talk in Action: Interactions, Identities and Institutions}. Oxford: Blackwell.Hsieh, E. 2007.  Interpreters as co-diagnosticians: overlapping roles and services between providers and interpreters},
% 	volume = {64},
% 	year = {2010}
% }
% 
% @incollection{Jefferson1978,
% 	address = {Antwerpen},
% 	author = {Jefferson, \citealt{Gail},
% 	booktitle = {\textit{Inequalities in Health Care for Migrants and Ethnic Minorities},},
% 	editor = {D. Ingleby, A. Chiarenza, W. Devillé and J. Kotsioni},
% 	pages = {82-97},
% 	publisher = {Garant},
% 	title = {}. Explanation of transcript notation. In: Schenkein, J. (Ed.), \textit{Studies in the organization of conversational interaction}. Academic Press, New York, pp. Xii-xvi.Lizana, T. 2012. Developing a migrant health policy for Catalonia},
% 	year = {1978}
% }
% 
% @incollection{Luhmann1984,
% 	address = {Amsterdam},
% 	author = {Luhmann, N},
% 	booktitle = {\textit{Proceedings of the First Forlì Conference on Interpreting Studies},},
% 	editor = {M. Viesni, and G. Gersam},
% 	pages = {160-169},
% 	publisher = {John Benjamins},
% 	title = {\textit{Soziale systeme}. Frankfurt a.M.: Suhrkamp.Meyer, B. 2002. Medical \isi{interpreting}. Some salient features},
% 	year = {1984}
% }
% 
% Merlini, R. 2009. “Seeking asylum and seeking identity in a mediated encounter: the projection of selves through discoursive practices.” Interpreting 11 (1): 57-92.
%
%
%
% Palfreyman, D. and M.al \citealt{Khalil2003}. A Funky Language for Teenzz to Use: Representing Gulf \ili{Arabic} in Instant Messaging\textit{.  Journal of Computer-Mediated Communication} 9 (1). Retreived April 24 2016 from: http://onlinelibrary.wiley.com/doi/10.1111/j.1083-6101.2003.tb00355.x/full
% 
% @book{Pearce1980,
% 	address = {New York},
% 	author = {Pearce B. and Cronen V.},
% 	publisher = {Praeger},
% 	title = {\textit{Communication, Action and Meaning}},
% 	year = {1980}
% }
% 
% @incollection{Penn2012,
% 	address = {Amsterdam/Philadelphia},
% 	author = {Penn, C. and Watermeyer, J},
% 	booktitle = {\textit{Coordinating participation in dialogue interpreting},},
% 	editor = {C. Baraldi and L. Gavioli},
% 	pages = {269-296},
% 	publisher = {John Benjamins},
% 	title = {Cultural brokerage and overcoming communication barriers: A case study for aphasia},
% 	year = {2012}
% }
% 
% @article{Pittarello2009,
% 	author = {Pittarello, S},
% 	journal = {\textit{The Interpreters' Newsletter}},
% 	pages = {59-90},
% 	title = {Interpreter mediated medical encounters in North Italy: Expectations, perceptions and practice},
% 	volume = {14},
% 	year = {2009}
% }
% 
% @incollection{Pöchhacker2008,
% 	address = {Amsterdam \& Philadelphia},
% 	author = {Pöchhacker, F},
% 	booktitle = {\textit{Crossing Borders in Community Interpreting. Definitions and dilemmas},},
% 	editor = {Carmen Valero-Garces and Anne Martin},
% 	pages = {9-26},
% 	publisher = {John Benjamins},
% 	title = {Interpreting as mediation},
% 	year = {2008}
% }
% 
% @article{Psathas1990,
% 	author = {Psathas, George, Anderson, Timothy},
% 	journal = {\textit{Semiotica}},
% 	number = {1/2},
% 	pages = {75--99},
% 	title = {The practices of transcription in conversation analysis},
% 	volume = {78},
% 	year = {1990}
% }
% 
% @book{Roberts2000,
% 	address = {Selected Papers from the Second International Conference on Interpreting in Legal, Health and Social Service Settings.} Amsterdam/Philadelphia},
% 	editor = {Roberts, R. P., Carr, S. F., Abraham, D. and Dufour, A.},
% 	publisher = {John Benjamins},
% 	title = {\textit{The Critical Link 2: Interpreters in the Community},
% 	year = {2000}
% }
% 
% @article{Robinson2001,
% 	author = {Robinson J},
% 	journal = {\textit{Social Science \& Medicine}},
% 	pages = {639-656},
% 	title = {Closing medical encounters: two physician practices and their implications for the expression of patients' unstated concerns},
% 	volume = {53},
% 	year = {2001}
% }
% 
% 
% \textstyleMaiuscoletto{\textup{Tebble, H. 1999.}} The tenor of consultant physicians: Implications for medical \isi{interpreting}. In \textstyleMaiuscoletto{\textup{I. Mason}} (ed.), \textit{Dialogue \isi{interpreting}.} \textit{The Translator} 5 (2): 179-200.
% 
% @incollection{Verrept2012,
% 	address = {Antwerpen},
% 	author = {Verrept, H},
% 	booktitle = {\textit{Inequalities in Health Care for Migrants and Ethnic Minorities},},
% 	editor = {D. Ingleby, A. Chiarenza, W. Devillé and J. Kotsioni},
% 	pages = {115-127},
% 	publisher = {Garant},
% 	title = {Notes on the employment of intercultural mediator and interpreters in health care},
% 	year = {2012}
% }
% 
% @book{Wadensjö1998,
% 	address = {London},
% 	author = {Wadensjö C.},
% 	publisher = {Longman},
% 	title = {\textit{Interpreting as interaction}},
% 	year = {1998}
% }
% 
% @book{Weigand2010,
% 	address = {Amsterdam},
% 	author = {Weigand E.},
% 	publisher = {John Benjamins},
% 	title = {\textit{Dialogue. The Mixed Game}},
% 	year = {2010}
% }
% 
% @book{Winslade2008,
% 	address = {San Francisco},
% 	author = {Winslade J. and Monk G.},
% 	publisher = {Jossey-Bass},
% 	title = {\textit{Practicing Narrative Mediation: Loosening the Grip of Conflict}},
% 	year = {2008}
% }
% 
